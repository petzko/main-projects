%% ****** Start of file apsguide4-1.tex ****** %
%%
%%   This file is part of the APS files in the REVTeX 4.1 distribution.
%%   Version 4.1r of REVTeX, August 2010.
%%
%%   Copyright (c) 2009, 2010 The American Physical Society.
%%
%%   See the REVTeX 4.1 README file for restrictions and more information.
%%
\documentclass[preprint,secnumarabic,amssymb, nobibnotes, aip, prd]{revtex4-1}
%\usepackage{acrofont}%NOTE: Comment out this line for the release version!
\newcommand{\revtex}{REV\TeX\ }
\newcommand{\classoption}[1]{\texttt{#1}}
\newcommand{\macro}[1]{\texttt{\textbackslash#1}}
\newcommand{\m}[1]{\macro{#1}}
\newcommand{\env}[1]{\texttt{#1}}
\setlength{\textheight}{9.5in}


\usepackage{amsmath,amsfonts,amssymb}
\usepackage{graphicx}
\usepackage[colorlinks=true, allcolors=blue]{hyperref}


\usepackage{color}
\usepackage[latin9]{inputenc}
\usepackage{mathrsfs,amsmath}
\usepackage{graphicx}%
\usepackage{float}
\usepackage{amsfonts}%
\usepackage[titletoc]{appendix}
\usepackage{amssymb}
\usepackage{braket}
\usepackage{bm}

\newcommand{\mb}[1]{\bm{#1}}
\usepackage[T1]{fontenc}

\def\Nabla{\bm{\nabla}}
\def\bm{\mathbf}
\def\curl{\Nabla\times}
\def\div{\Nabla\cdot}
\def\lap{\Delta}
\def\vlap{\Delta}
\def\x{\hat{e}_{x}}
\def\y{\hat{e}_{y}}
\def\z{\hat{e}_{z}}
\def\p{\partial}
\def\h{\hat}
\def\h{\hat}
\def\tw{\tilde{\omega}}
\def\gm{\gamma}
\def\om{\omega}
\def\OM{\Omega}
\def\GM{\Gamma}
\def\dw{\delta\omega}
\def\dth{\Delta\theta}
\def\dk{\delta k}
\def\Hdth{\frac{\dth}{2}} %half Delta Theta
\def\P{\hat{\pi}_+}
\def\M{\hat{\pi}_-}


\DeclareMathOperator{\sech}{sech}
\DeclareMathOperator{\csch}{csch}
\DeclareMathOperator{\arcsec}{arcsec}
\DeclareMathOperator{\arccot}{arcCot}
\DeclareMathOperator{\arccsc}{arcCsc}
\DeclareMathOperator{\arccosh}{arcCosh}
\DeclareMathOperator{\arcsinh}{arcsinh}
\DeclareMathOperator{\arctanh}{arctanh}
\DeclareMathOperator{\arcsech}{arcsech}
\DeclareMathOperator{\arccsch}{arcCsch}
\DeclareMathOperator{\arccoth}{arcCoth} 

%\usepackage[font=small]{caption}
\newcommand{\includegraphicsXL}[1]{\includegraphics[width = 0.4\textwidth]{#1}}
%\captionsetup{width=.45\textwidth}




\bibliographystyle{ieeetr}
\begin{document}

\section{Derivation of the two level field propagation equations}

Note that, for convenience,  I have changed the notation from the powerpoint slides and used the symbol $f(x,t)$ to denote the electric field \emph{envelope}, whereas the letter $E$ is reserved for the ELECTRIC FILED.


For active atoms embedded in a passive medium the total polarization comprises of two terms
\begin{equation}
P_{tot} = P+P_0 = P+\epsilon_0\chi_0E_z
\end{equation}
where the first term is the polarization contributed by the active atoms and $P_0$ is the background polarization as a linear reaction to the EM field due to the constant background susceptibility $\chi_0$. The background refractive index can be written as 
$n_{THz}^2 = 1+\chi_0$ which yields the wave equation 
\begin{align}
\label{eq:waveqn}
\underbrace{\left [\frac{c^2}{n_{THz}^2} \frac{\p^2}{\p x^2} -\frac{\p^2}{\p t^2} \right ] E_z}_{LHS} =\overbrace{\frac{1}{\epsilon_0 n_{THz}^2}\frac{\p^2}{\p t^2}P}^{RHS}  
\end{align}

The incident light is assumed linear polarized in z-direction and slowly varying:
\begin{equation}
\label{eq:ringansatz-field}
E_z(x,t) =\frac{1}{2}E(x,t)e^{i(k_0x-\omega_0t)}+c.c.,  
\end{equation}
where $E(x,t)$ is the slowly varying envelope, $\omega_0$ is the central frequency and we also \emph{assume} that $k_0=\omega_0 n_{THz}/c$ is the corresponding wave number. 
The polarization $P$ takes the form 
\begin{equation}
P(x,t) =  -N\Gamma\mu (\rho_{12}+\rho_{21}),  
\end{equation}
which is obtained by calculating the expectation value of the dipole moment operator in the two level system. $N\Gamma$ is the 
"effective" active carriers' volume density and $\mu$ is the optical transition's dipole moment (in units of $C\cdot m$). We have taken a minus sign due to the fact that in our convention $\mu = e\bra{2}\hat{z}\ket{1}$, where $e$ is the \emph{positive} elementary charge. 

We decompose $\rho_{21}=\eta_{21}e^{i(k_0x-\omega_0t)}$, perform the differentiation in the wave equation and use that, within the slowly varying envelope approximation (SVEA), it holds
\begin{align}
\label{eq:SVEA}
\left |\frac{\p^2 E}{\p t^2}\right | \ll \omega_0\left|\frac{\p E}{\p t}\right| \quad \text{, } \left |\frac{\p^2 E}{\p x^2}\right | \ll k_0\left|\frac{\p E}{\p x}\right| \quad \text{, } \left |\frac{\p^2 \eta}{\p t^2}\right | \ll \omega_0^2 \left| \eta_{21}\right| \quad \text{, } \omega_0 \left| \frac{\p \eta_{21}}{\p t}\right| \ll  \omega_0^2\left|\eta_{21}\right|.
\end{align}
Expanding Eq. (\ref{eq:waveqn}) we obtain
\begin{align}
LHS &= \frac{1}{2} \frac{c^2}{n_{THz}^2} \left(\frac{\p^2 E}{\p x^2} +2ik_0 \frac{\p E}{\p x} -k_0^2 E\right)e^{i(k_0x-\omega_0t)} - \frac{1}{2}\left(\frac{\p^2 E}{\p t^2} -2i\omega_0 \frac{\p E}{\p t} -\omega_0^2 E\right)e^{i(k_0x-\omega_0t)} + c.c. \nonumber \\
RHS &= -\frac{N\Gamma\mu}{\epsilon_0 n_{THz}^2} \left(\frac{\p^2 \eta_{21}}{\p t^2} -2i\omega_0 \frac{\p \eta_{21}}{\p t} -\omega_0^2. \eta_{21}\right)e^{i(k_0x-\omega_0t)}+c.c.
\end{align}
Now, if we apply the SVEA approximation, i.e. Eq. (\ref{eq:SVEA}), and compare the coefficients in front of the exponents we get
\begin{align}
\label{eq:almostdone}
\frac{c^2}{n_{THz}^2}ik_0 \frac{\p E}{\p x}+i\omega_0 \frac{\p E}{\p t} = \frac{N\Gamma\mu\omega_0^2}{\epsilon_0 n_{THz}^2}\eta_{21}.
\end{align}
Lastly, we divide Eq. (\ref{eq:almostdone}) by $i\omega_0$ and acknowledge that we have set $k_0 = \omega_0 n_{THz}/c$ to get
\begin{align}
\label{eq:almostdone2}
\frac{c}{n_{THz}} \frac{\p E}{\p x}+ \frac{\p E}{\p t} = -i\frac{N\Gamma\mu\omega_0}{\epsilon_0 n_{THz}^2}\eta_{21},
\end{align}
or equivalently
\begin{align}
\label{eq:almostdone3}
\frac{\p E}{\p x}+ \frac{n_{THz}}{c} \frac{\p E}{\p t} = -i\frac{N\Gamma\mu\omega_0}{\epsilon_0 c n_{THz}}\eta_{21}.
\end{align}

\section{Maxwell-Bloch equations}
\subsection{Ring Cavity}
For a ring cavity we have only unidirectional propagation and thus the system of equations is very simple.  
\begin{align}
& \frac{\p E}{\p x}+ \frac{n_{THz}}{c} \frac{\p E}{\p t} = -i\frac{N\Gamma\mu\omega_0}{\epsilon_0 c n_{THz}}\eta_{21}-\frac{a}{2}E.
\\
& \partial_t {\rho_{2,2}(t)} = i\frac{ \mu }{2 \hbar} \Big ( E^* \eta_{21} - E\eta_{21}^* \Big) -\frac{\rho_{2,2}-\rho_{2,2}^{eq}}{T_1}\text{  ,} \label{eq:inversion1}
\\
& \partial_t {\rho_{1,1}(t)} = - i\frac{ \mu }{2 \hbar} \Big ( E^* \eta_{21} - E\eta_{21}^* \Big) -\frac{\rho_{1,1}-\rho_{1,1}^{eq}}{T_1} \label{eq:inversion2}
\\
& d_t \eta_{21} = -i(\omega_{21}-\omega_0)\eta_{21}+i\frac{ \mu}{ 2 \hbar} E(\rho_{22}-\rho_{11}) - \frac{\eta_{21}}{T2} \text{  ,}
\end{align}
where $a = a_w+a_m$ denotes the distributed waveguide and mirror POWER losses. The boundary conditions are naturally $E(x=0,t)=E(x=L,t)$, where $L$ is the cavity length.

\subsection{Fabry-Perot cavity}

For a Fabry-Perot cavity we need to allow for both forward and backward travelling components of the field. The interference between those counter-propagating waves will induce intensity grating, which in turn will induce a population grating, commonly referred to as the spatial hole burning (SHB) effect. Due to the relatively slow carrier diffusion as compared to the life-times in QCLs, SHB cannot be neglected. Within the rotating wave approximation SHB is usually modelled only to first order by making the ansatz 
\begin{align}
\rho_{jj}(t) = \rho_{jj}^{0} + \rho_{jj}^{+}e^{2ik_0x}+\rho_{jj}^{-}e^{-2ik_0x}, 
\end{align}
where $\rho_{jj}^{0}$ can be interpreted as population averages, and $\rho_{jj}^{+}=(\rho_{jj}^{-})^{\ast}$ as the amplitudes of the inversion grating. Naturally for the electric field and the polarization the following ansatz also applies 
\begin{align}
E_{z}(x,t) &=\frac{1}{2}\left( E_{+}(x,t)e^{\mathrm{i}(k_{0}x-\omega_{0}t)} +E_{-}(x,t)e^{-\mathrm{i}(k_{0}x+\omega_{0}t)}  +c.c\right)  , \label{eq:FPansatz-field} \\
\rho_{21}(x,t)  &  =\eta_{21}^{+}(x,t)e^{  \mathrm{i}(k_{0}x-\omega_{0}t)}  +\eta_{21}^{-}(x,t)e^{-\mathrm{i}(k_{0}x+\omega_{0}t)}  ,\label{eq:21ansatz}
\end{align}

Then the density matrix equations (after some approximations) become 
\begin{align}
\frac{d\rho_{22}^{0}}{dt} &= \mathrm{i}\frac{\mu}{2\hslash}\left(  E_{-}^{\ast}\eta_{21}^{-}+E_{+}^{\ast}\eta_{21}^{+}-c.c.\right) -\frac{\rho_{2,2}^{0}-\rho_{2,2}^{eq}}{T_1} \\
\frac{d\rho_{22}^{+}}{dt} &= \mathrm{i}\frac{\mu}{2\hslash}\left[  E_{-}^{\ast}\eta_{21}^{+}-E_{+}(\eta_{21}^{-})^{\ast}\right] - \left( \frac{1}{T_1}+4k_{0}^{2}D\right)  \rho_{22}^{+},\label{eq:rtpop3grating}\\
\frac{d\rho_{11}^{0}}{dt} &=-\mathrm{i}\frac{\mu}{2\hslash}\left(E_{-}^{\ast}\eta_{21}^{-}+E_{+}^{\ast}\eta_{21}^{+}-c.c.\right) -\frac{\rho_{1,1}^{0}-\rho_{1,1}^{eq}}{T_1} \\
\frac{d\rho_{11}^{+}}{dt} &=-\mathrm{i}\frac{\mu}{2\hslash}\left[E_{-}^{\ast}\eta_{21}^{+}-E_{+}(\eta_{21}^{-})^{\ast}\right] -\left(  \frac{1}{T_1}+4k_{0}^{2}D\right) \rho_{11}^{+}, \label{eq:rtpop2grating} \\
\frac{d\eta_{21}^{\pm}}{dt} & = -\mathrm{i}\left(  \omega_{21}-\omega_{0}\right) \eta_{21}^{\pm}+\mathrm{i}\frac{\mu}{2\hslash}\left[  E_{\pm}(\rho_{22}^{0}-\rho_{11}^{0})+E_{\mp}(\rho_{22}^{\pm}-\rho_{11}^{\pm})\right]-\frac{1}{T_2}\eta_{21}^{\pm},
\end{align}
where D is the diffusion constant, $k_0$ is the wave number and the $^{\ast}$ denotes the complex conjugate.

Finally the propagation equations are
\begin{equation}
\frac{n_{THz}}{c}\frac{\partial E_{\pm}}{\partial {t}}\pm\frac{\partial E_{\pm}}{\partial {x}}=-i\frac{N\Gamma\mu\omega_0}{\epsilon_0 c n_{THz}}\eta_{21}^{\pm}-\frac{a}{2}E_{\pm
}, \label{eq:rtwave}%
\end{equation}	
With the "reflecting" boundary conditions $E_+(x=0,t)=E_-(x=0,t)$ and $E_-(x=L,t)=E_+(x=L,t)$, where again $L$ is the cavity length, and $a$ denotes the distributed power losses. 


\section{Threshold gain}

We first examine the steady state solution of the Maxwell-Bloch equations at threshold. At and below the threshold pumping, the electric field is zero, so from Eq.~(\ref{eq:inversion1}) and (\ref{eq:inversion2}), we get the steady state solution $\bar{\Delta}_0 = \Delta_{0}^{eq} = \Delta_0^{\text{th}}$. At threshold, the peak of the gain, attained at some angular frequency $\omega$ will exactly balance the losses, so the threshold condition tells us that
\begin{equation}
\label{eq:gain-loss-threshold}
\frac{N\Gamma\mu^2\omega_0}{2 \epsilon_0 c \hbar n_{THz}}\times\frac{\gamma_{21}}{\gamma_{21}^2+(\omega-\omega_{21})^2}\times\bar{\Delta}_0 =\frac{1}{2}\sigma_\omega N \Delta_0^{\text{th}} = \frac{a}{2},
\end{equation}
where $\gamma_{21} = T_2^{-1}$ is the dephasing rate and $\sigma_\omega$ is the gain cross-section at that particular frequency, given by
\begin{equation}
\label{eq:cross-section}
\sigma_\omega = \frac{\Gamma \mu^2\omega_{0}}{\hbar\epsilon_0n_0c}\times\frac{\gamma_{21}}{\gamma_{21}^2+(\omega-\omega_{21})^2}.
\end{equation}
This gives us the threshold population inversion as 
\begin{equation}
\label{eq:threshold-inversion}
\Delta_0^{\text{th}} = \frac{a}{\sigma_{\omega}N}, 
\end{equation}
where the primary lasing frequency, i.e. $\omega$, will generally be the cavity mode yielding the largest possible value of $\sigma_\omega$ and thus smallest possible value for the inversion $\Delta_{0}^{\text{th}}$. Since it is reasonable to assume that $\omega \approx \omega_{21}\approx\omega_0$, the threshold inversion is given by the formula $\Delta_0^{\text{th}} = a \times \epsilon_0 c \hbar n_{THz}/\left [N\Gamma\mu^2\omega_0 T_2\right ]$.

We now rewrite the equation for the population inversion in terms of the "pump" strength parameter $p=\Delta_0^{\text{eq}}/\Delta_0^{\text{th}}$

\section{Haus's theory of passive mode locking with a fast saturable absorber}
\label{sec:haustheory}

Due to the relatively low saturation intensity in quantum cascade lasers, the classical theory of mode locking with a fast saturable absorber will not apply, since the gain will also saturate. Below we will write down a master equation modelling the passive mode locking  with a FAST saturable absorber and a saturable gain. 

Equation for the inversion upon passage of a pulse 
\begin{align}
\dot \Delta = - \kappa_g^2T_{2g} |E|^2 \Delta  \nonumber \\
			&= - \frac{1}{T_{1g}}\frac{|E|^2}{W_{sat}^G}\Delta, 
\end{align}
where $W_{sat}^G = 1/[\kappa_g^2T_{1g}T_{2g}]$ is the saturation electric field squared of the gain medium. The solution of the above equation is 
\begin{equation}
\Delta(t) = \Delta_{init}\exp\{-\frac{1}{T_{1g}} \int_{-T_R/2}^{t} \frac{|E(\tau)|^2}{W_{sat}^G}d\tau\}
\end{equation}
Thus the gain $g(t) = \sigma_\omega^g N_g \Delta(t)$ also follows the same exponential dependence.  

For a FAST saturable absorber we have~\cite{haus1975theory}
\begin{equation}
l(t) = (a+q_0) -\gamma |E|^2,
\end{equation}
where $a$ denotes the distributed losses, $q_0 = \sigma_\omega^a N_a$ is the unsaturated loss in the SA, and $\gamma = q_0/W_{sat}^A$ is the so called self-amplitude modulation (SAM) parameter.



The master equation then reads
\begin{align}
\label{eq:master-equation}
T_R \frac{\p E(T,t)}{\p T} &= \left[g(t) - l(t) +\Delta_f \frac{\p^2}{\p t^2}\right]E(T,t) \nonumber \\
T_R \frac{\p E(T,t)}{\p T} &= \left[ g_{init}\exp\left \{-\frac{1}{T_{1g}} \int_{-T_R/2}^{t} \frac{|E(\tau)|^2}{W_{sat}^G}d\tau \right \} - (a+q_0)  +D_f \frac{\p^2}{\p t^2} +\gamma |E|^2\right]E(T,t).
\end{align}
Let us expand the exponent up to second order
\begin{align}
g_{init}(t) \approx  g_{init}\left [ 1-\frac{1}{T_{1g}} \int_{-T_R/2}^{t} \frac{|E(\tau)|^2}{W_{sat}^G}d\tau +\frac{1}{2} \left(\frac{1}{T_{1g}} \int_{-T_R/2}^{t} \frac{|E(\tau)|^2}{W_{sat}^G}d\tau\right)^2 \right ].
\end{align}

Now we can make the following ansatz for the solution $E(t) = E_0 \sech(\frac{t}{\tau} +\alpha \frac{T}{T_R})$ and use the expressions~\cite{haus1975theoryslow}
\begin{align}
\frac{1}{T_{1g}W_{sat}^G}\int_{-T_R/2}^{t} |E(T,\tau)|^2d\tau &= \frac{|E_0|^2\tau}{T_{1g}W_{sat}^G} [1+\tanh(\frac{t}{\tau}+\alpha \frac{T}{T_R})], \\ 
(\frac{1}{T_{1g}W_{sat}^G})^2\left (\int_{-T_R/2}^{t} |E(T,\tau)|^2d\tau \right)^2 &=  \left(\frac{|E_0|^2\tau}{T_{1g}W_{sat}^G}\right)^2 \left(2+2\tanh(\frac{t}{\tau}+\alpha \frac{T}{T_R}) - \sech^2(\frac{t}{\tau}+\alpha \frac{T}{T_R})\right), \\
T_R\frac{\p E}{\p T} &= -\alpha \tanh(\frac{t}{\tau}+\alpha \frac{T}{T_R}) E(T,t), \\
D_f\frac{\p^2 E}{\p t^2} &= \frac{D_f}{\tau^2}\left( 1-2\sech^2(\frac{t}{\tau}+\alpha \frac{T}{T_R}) \right) E(T,t).
\end{align}

Now combine in the master equation 
\begin{align}
\label{eq:master-equation2}
 -\alpha \tanh(\frac{t}{\tau}+\alpha \frac{T}{T_R}) E(T,t) &=  \{ g_{init}(1 - \frac{|E_0|^2\tau}{T_{1g}W_{sat}^G} [1+\tanh(\frac{t}{\tau}+\alpha \frac{T}{T_R})]  \nonumber \\
 &  +\frac{1}{2} \left(\frac{|E_0|^2\tau}{T_{1g}W_{sat}^G}\right)^2 \left(2+2\tanh(\frac{t}{\tau}+\alpha \frac{T}{T_R}) - \sech^2(\frac{t}{\tau}+\alpha \frac{T}{T_R})\right)  \nonumber \\
 &  -(a+q_0)  + \frac{D_f}{\tau^2}\left( 1-2\sech^2(\frac{t}{\tau}+\alpha \frac{T}{T_R}) \right) + \gamma |E|^2  \}E(T,t).
\end{align}
Or 
\begin{align}
\label{eq:master-equation3}
-\alpha \tanh(\frac{t}{\tau}+\alpha \frac{T}{T_R}) &=  \{ g_{init}(1 - \frac{|E_0|^2\tau}{T_{1g}W_{sat}^G} [1+\tanh(\frac{t}{\tau}+\alpha \frac{T}{T_R})]  \nonumber \\
&  +\frac{1}{2} \left(\frac{|E_0|^2\tau}{T_{1g}W_{sat}^G}\right)^2 \left(2+2\tanh(\frac{t}{\tau}+\alpha \frac{T}{T_R}) - \sech^2(\frac{t}{\tau}+\alpha \frac{T}{T_R})\right)  \nonumber \\
&  -(a+q_0)  + \frac{D_f}{\tau^2}\left( 1-2\sech^2(\frac{t}{\tau}+\alpha \frac{T}{T_R}) \right) + \gamma |E|^2  \}.
\end{align}

Now we arrange the coefficients. If we put all terms terms before $\tanh $ together on the right hand side we get:
\begin{align}
\label{eq:coefs1}
\left[\alpha -g_{init}\frac{|E_0|^2\tau}{T_{1g}W_{sat}^G} +g_{init}\left(\frac{|E_0|^2\tau}{T_{1g}W_{sat}^G}\right)^2 \right]\tanh(\frac{t}{\tau}+\alpha \frac{T}{T_R}).
\end{align}
If we put all terms terms before $\sech^2 $ together on the right hand side we get:
\begin{align}
\label{eq:coefs2}
\left[ -g_{int}\frac{1}{2}\left(\frac{|E_0|^2\tau}{T_{1g}W_{sat}^G}\right)^2 - 2\frac{D_f}{\tau^2} +\gamma|E_0|^2\right]\sech(\frac{t}{\tau}+\alpha \frac{T}{T_R}).
\end{align}
If we put all constant terms together on the right hand side we get:
\begin{align}
\label{eq:coefs3}
g_{init}\left[1-\frac{|E_0|^2\tau}{T_{1g}W_{sat}^G}+\left(\frac{|E_0|^2\tau}{T_{1g}W_{sat}^G}\right)^2 \right] -(a+q_0) - \frac{D_f}{\tau^2}.
\end{align}

Now the parametric regimes where the solution will exist will satisfy that all those expressions are identically zero! We try a solution with $\alpha = 0$. From Eq. (\ref{eq:coefs1}) we get that
\begin{equation}
\frac{|E_0|^2\tau}{T_{1g}W_{sat}^G}=1,
\end{equation}
which when we plug into Eq. (\ref{eq:coefs3}) yields
\begin{equation}
\frac{D_f}{\tau^2} = g_{init} - a-q_0. 
\end{equation}
and from Eq. \ref{eq:coefs3} we finally have
\begin{equation}
\gamma|E_0|^2 = \frac{5}{2}g_{init}-a-q_0.
\end{equation}

ToDo: Examine the solution and especially the stability condition! 



\section{Some notes/thoughts on the paper}
\subsection{}
	It is important to realize that what matters is the gain recovery time ABOVE THRESHOLD and not to the steady state value w0 of the inversion. This will have a large impact on the mode locking, since for a fixed T1 parameter, the gain recovery time will be shorter if the threshold is LOWER and longer if the threshold is HIGHER. So for ideal results one might want to bias the cavity only slightly above threshold.
\subsection{}
	Note however, that the above argument might not be exactly true. One needs to also consider the losses upon propagating inside the absorber! Assuming perfectly inverted absorber of length $L_a$, without distributed losses, then the condition in Eq. (\ref{eq:threshold-inversion}) becomes
	\begin{align}
	\label{eq:threshold-inversion2}
	&\sigma_{\omega}^{g}N_g\Delta_0^{\text{th}}L_g = aL_g+ \sigma_{\omega}^a N_a L_a, \nonumber \\
	& \Leftrightarrow \nonumber \\
	&\Delta_0^{\text{th}} = \frac{a}{\sigma_{\omega}^{g}N_g} + \frac{\sigma_{\omega}^{a}N_a L_a}{\sigma_{\omega}^{g}N_g L_g}, \nonumber \\
	& \Leftrightarrow \nonumber \\
	&\Delta_0^{\text{th}} = \frac{a}{\sigma_{\omega}^{g}N_g} +  \frac{G_a}{G_g},
	\end{align}
	where subscript/superscript indices $g$/$a$ denote the respective quantities of the gain/absorber section and $G_j = \sigma_{\omega}^{j}N_j L_j$. From Eq. (\ref{eq:threshold-inversion2}), we immediately see that 
	if $G_a>G_g$ then $\Delta_0^{th} > 0$, and no lasing will start whatsoever. 
\subsection{}
	It seems that the mode locking mechanism is very sensitive to the dephasing time $T_{2a}$ of the absorber precisely of that reason! When you DECREASE the pure dephasing time by $T_{2a}\rightarrow T_{2a}/2$, then the absorption cross section $\sigma_\omega^{a} \rightarrow \sigma_\omega^{a}/2$ which LOWERS the threshold $\Delta_{0}^{th}$ due to Eq. (\ref{eq:threshold-inversion2}) and thus broadens the pulse. I expect that the parameter $p$ will thus play a vital role in the dynamics. IDEALLY always keep $p \approx 1$!  
\subsection{}
Actually we can straight-forwardly derive an expression for the gain recovery time. 

Assume that the pulse has just passed the atom and at time $t=t_0$, perturbed the system to inversion $\Delta_0(t_0)$. What will the gain recovery be then? Well it will be the time it takes for the population inversion to recover above THRESHOLD value $\Delta_{0}^{th}$. Immediately after passage of the pulse, the population inversion obeys the following simple ODE
\begin{equation}
\dot\Delta_0(t) = -\frac{\Delta_0(t) -\Delta_0^{eq}}{T_{1g}},
\end{equation}    
which has the solution 
\begin{eqnarray}
\Delta_0(t) = \big(\Delta_0(t_0) -\Delta_0^{eq}\big)e^{-\frac{t-t_0}{T_{1g}}}+\Delta_0^{eq}.
\end{eqnarray}
Now substituting $\Delta_0^{eq} = p\Delta_0^{(th)}$ and requiring that at $t=t_1$ the gain has recovered to the threshold value $\Delta_0^{th}$ we obtain an equation for the gain recovery time $\Delta T = t_1-t_0$ 
\begin{equation}
\Delta_{0}^{th} = \big(\Delta_0(t_0) - p\Delta_0^{th}\big)e^{-\frac{\Delta T}{T_{1g}}}+p\Delta_0^{th}.
\end{equation}
with the gain recovery $\Delta T $ given by
\begin{equation}
\Delta T = T_{1g} \ln\frac{p\Delta_0^{th}-\Delta_0(t_0)}{\Delta_0^{th}(p-1)}.
\end{equation}
One can naively assume that if $p\to1+$ the gain recovery time will be infinite, which is long enough for mode locking. This is obviously not the case since we have not written down how would the saturated gain $\Delta_0(t_0)$ depend on the pump strength $p$. Intuitively, the closer the $p$ is to threshold, the weaker the saturation and hence the closer this value to $p\Delta_0^{th}$ which will render gain recovery time of 0. So we see that close to threshold there is a very peculiar dynamics which needs to be investigated in detail. To do so, we need to estimate the saturation strength as a function of the pulse intensity, which on the other hand will be a function of the pump power and possibly other system parameters. For this, we need a comprehensive model for the mode locking dynamics, possibly similar to the celebrated Haus master equation, but one which includes both the effects of a FAST saturable absorber and a SATURABLE gain, which is done in Sec. \ref{sec:haustheory}. 

\subsection{}
To investigate the dependence of the maximal saturation $\Delta_0(t_0)$ onto the different parameters we will address the equation
\begin{equation}
\label{eq:inversion3}
\dot\Delta_0(t) = -\frac{\mu^2}{\hbar^2}T_{2g} |E|^2\Delta_0(t) -\frac{\Delta_0-\Delta_0^{eq}}{T_{1g}}.
\end{equation}
Assuming instantaneous response of the inversion to the applied field, the population inversion will reach it's minimum as a function of time, at a point where $\dot\Delta_0 = 0$. This readily gives us the solution
\begin{equation}
\label{eq:saturation}
\Delta_0(t_0) =  \frac{ p\Delta_0^{th}}{1+\kappa_g^2T_{1g}T_{2g}|E|^2} = \frac{ p\Delta_0^{th}}{1+|E|^2/W_{sat}^G} ,
\end{equation} 
where we have used the same notation as in Sec. \ref{sec:haustheory}.


First of all, we notice that $\Delta_0(t_0) >0$ for all $p$. This means that for a fixed pump parameter $p$, the maximum value of the gain recovery is $\Delta T= T_{1g}\times\ln p/(p-1)$. Which can be arbitrarily large for $p\to 1$. There is more to it, as the saturation will not necessarily be 0 but larger. This will depend on the coupling strength $\kappa_g$, the saturation intensity of the gain medium $I_{sat}\propto 1/T_{1g}T_{2g}$ and moreover on the amplitude of the field $|E|$.

A formula, more revealing to the nature of the gain recovery dynamics, in a pumped two level system is thus the following
\begin{align}
\Delta T = T_{1g} \ln\left[\frac{p}{\left(p-1\right)}\times\frac{|E|^2/W_{sat}^G}{\left (1+|E|^2/W_{sat}^G\right)}\right].
\end{align}
%Following Haus's notation and solution \cite{haus1975theory}, we know that successful passive mode locking with a fast saturable absorber would produce a hyperbolic secant pulse satisfying the following relations
%
% \begin{align}
% v(t) &= v_0 \sech(t/\tau_p) , \\
% \frac{qv_0^2}{P_A} &= \frac{2g}{\omega_L^2\tau_p^2} , \\
% 1+q-g &= \frac{g}{\omega_L^2\tau_p^2}.
% \end{align}
% Below I present a table with the respective quantities and their relation to our model
% \begin{table}
%	\begin{tabular}{c|c|c|c}
%		\hline
%	\textbf{Haus-Symbol} & \textbf{Meaning} & \textbf{Units} & \textbf{Relation to our model} \\	
%	\hline
%	$A_A$ & Effective absorber area & $ m^2 $ & $A_A$ \\
%	$v_0$ & $\sqrt{\text{generated power}}$  & Watts & $\sqrt{\epsilon_0n_0 c|E_0|^2 A_A}$\\
%	$P_A$ & Saturation Power & Watts & $\frac{\hbar \omega_0 }{\sigma_\omega^a T_{1a}}\times A_A$ \\
%	$q$ & the small signal (normalized) inverse Q of the SA & unit less & $\frac{\sigma_\omega^a L_A A_A N_A}{aL_G}$
%	\end{tabular}
% \end{table}

\bibliography{bib_resources.bib}
\end{document}

