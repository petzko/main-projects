%% ****** Start of file apsguide4-1.tex ****** %
%%
%%   This file is part of the APS files in the REVTeX 4.1 distribution.
%%   Version 4.1r of REVTeX, August 2010.
%%
%%   Copyright (c) 2009, 2010 The American Physical Society.
%%
%%   See the REVTeX 4.1 README file for restrictions and more information.
%%
\documentclass[preprint,secnumarabic,amssymb, nobibnotes, aip, prd]{revtex4-1}
%\usepackage{acrofont}%NOTE: Comment out this line for the release version!
\newcommand{\revtex}{REV\TeX\ }
\newcommand{\classoption}[1]{\texttt{#1}}
\newcommand{\macro}[1]{\texttt{\textbackslash#1}}
\newcommand{\m}[1]{\macro{#1}}
\newcommand{\env}[1]{\texttt{#1}}
\setlength{\textheight}{9.5in}


\usepackage{amsmath,amsfonts,amssymb}
\usepackage{graphicx}
\usepackage[colorlinks=true, allcolors=blue]{hyperref}


\usepackage{color}
\usepackage[latin9]{inputenc}
\usepackage{mathrsfs,amsmath}
\usepackage{graphicx}%
\usepackage{float}
\usepackage{amsfonts}%
\usepackage[titletoc]{appendix}
\usepackage{amssymb}
\usepackage{braket}
\usepackage{bm}

\newcommand{\mb}[1]{\bm{#1}}
\usepackage[T1]{fontenc}

\def\Nabla{\bm{\nabla}}
\def\bm{\mathbf}
\def\curl{\Nabla\times}
\def\div{\Nabla\cdot}
\def\lap{\Delta}
\def\vlap{\Delta}
\def\x{\hat{e}_{x}}
\def\y{\hat{e}_{y}}
\def\z{\hat{e}_{z}}
\def\p{\partial}
\def\h{\hat}
\def\h{\hat}
\def\tw{\tilde{\omega}}
\def\gm{\gamma}
\def\om{\omega}
\def\OM{\Omega}
\def\GM{\Gamma}
\def\dw{\delta\omega}
\def\dth{\Delta\theta}
\def\dk{\delta k}
\def\Hdth{\frac{\dth}{2}} %half Delta Theta
\def\P{\hat{\pi}_+}
\def\M{\hat{\pi}_-}

%\usepackage[font=small]{caption}
\newcommand{\includegraphicsXL}[1]{\includegraphics[width = 0.4\textwidth]{#1}}
%\captionsetup{width=.45\textwidth}




\bibliographystyle{ieeetr}
\begin{document}

\section{Derivation of the two level field propagation equations}

Note that, for convenience,  I have changed the notation from the powerpoint slides and used the symbol $f(x,t)$ to denote the electric field \emph{envelope}, whereas the letter $E$ is reserved for the ELECTRIC FILED.


For active atoms embedded in a passive medium the total polarization comprises of two terms
\begin{equation}
P_{tot} = P+P_0 = P+\epsilon_0\chi_0E
\end{equation}
where the first term is the polarization contributed by the active atoms and $P_0$ is the background polarization as a linear reaction to the EM field due to the constant background susceptibility $\chi_0$. The background refractive index can be written as 
$n_{THz}^2 = 1+\chi_0$ which yields the wave equation 
\begin{align}
\label{eq:waveqn}
\underbrace{\left [\frac{c^2}{n_{THz}^2} \frac{\p^2}{\p x^2} -\frac{\p^2}{\p t^2} \right ] E}_{LHS} =\overbrace{\frac{1}{\epsilon_0 n_{THz}^2}\frac{\p^2}{\p t^2}P}^{RHS}  
\end{align}

The incident light is assumed linear polarized in z-direction and slowly varying:
\begin{equation}
E(x,t) =\frac{1}{2}f(x,t)e^{i(k_0x-\omega_0t)}+c.c.,  
\end{equation}
where $f(x,t)$ is the slowly varying envelope, $\omega_0$ is the central frequency and we also \emph{assume} that $k_0=\omega_0 n_{THz}/c$ is the corresponding wave number. 
The polarization $P$ takes the form 
\begin{equation}
P(x,t) =  -N\Gamma\mu (\rho_{12}+\rho_{21}),  
\end{equation}
which is obtained by calculating the expectation value of the dipole moment operator in the two level system. $N\Gamma$ is the 
"effective" active carriers' volume density and $\mu$ is the optical transition's dipole moment (in units of $C\cdot m$). We have taken a minus sign due to the fact that in our convention $\mu = e\bra{2}\hat{z}\ket{1}$, where $e$ is the \emph{positive} elementary charge. 

We decompose $\rho_{21}=\eta_{21}e^{i(k_0x-\omega_0t)}$, perform the differentiation in the wave equation and use that, within the slowly varying envelope approximation (SVEA), it holds
\begin{align}
\label{eq:SVEA}
\left |\frac{\p^2 f}{\p t^2}\right | \ll \omega_0\left|\frac{\p f}{\p t}\right| \quad \text{, } \left |\frac{\p^2 f}{\p x^2}\right | \ll k_0\left|\frac{\p f}{\p x}\right| \quad \text{, } \left |\frac{\p^2 \eta}{\p t^2}\right | \ll \omega_0^2 \left| \eta_{21}\right| \quad \text{, } \omega_0 \left| \frac{\p \eta_{21}}{\p t}\right| \ll  \omega_0^2\left|\eta_{21}\right|.
\end{align}
Expanding Eq. (\ref{eq:waveqn}) we obtain
\begin{align}
LHS &= \frac{1}{2} \frac{c^2}{n_{THz}^2} \left(\frac{\p^2 f}{\p x^2} +2ik_0 \frac{\p f}{\p x} -k_0^2 f\right)e^{i(k_0x-\omega_0t)} - \frac{1}{2}\left(\frac{\p^2 f}{\p t^2} -2i\omega_0 \frac{\p f}{\p t} -\omega_0^2 f\right)e^{i(k_0x-\omega_0t)} + c.c. \nonumber \\
RHS &= -\frac{N\Gamma\mu}{\epsilon_0 n_{THz}^2} \left(\frac{\p^2 \eta_{21}}{\p t^2} -2i\omega_0 \frac{\p \eta_{21}}{\p t} -\omega_0^2. \eta_{21}\right)e^{i(k_0x-\omega_0t)}+c.c.
\end{align}
Now, if we apply the SVEA approximation, i.e. Eq. (\ref{eq:SVEA}), and compare the coefficients in front of the exponents we get
\begin{align}
\label{eq:almostdone}
\frac{c^2}{n_{THz}^2}ik_0 \frac{\p f}{\p x}+i\omega_0 \frac{\p f}{\p t} = \frac{N\Gamma\mu\omega_0^2}{\epsilon_0 n_{THz}^2}\eta_{21}.
\end{align}
Lastly, we divide Eq. (\ref{eq:almostdone}) by $i\omega_0$ and acknowledge that we have set $k_0 = \omega_0 n_{THz}/c$ to get
\begin{align}
\label{eq:almostdone2}
\frac{c}{n_{THz}} \frac{\p f}{\p x}+ \frac{\p f}{\p t} = -i\frac{N\Gamma\mu\omega_0}{\epsilon_0 n_{THz}^2}\eta_{21},
\end{align}
or equivalently
\begin{align}
\label{eq:almostdone3}
\frac{\p f}{\p x}+ \frac{n_{THz}}{c} \frac{\p f}{\p t} = -i\frac{N\Gamma\mu\omega_0}{\epsilon_0 c n_{THz}}\eta_{21}.
\end{align}

\section{Discretization}

Equation (\ref{eq:almostdone3}) is a first order hyperbolic partial differential equation which, although might seem easy at first sight, is non-trivial to solve numerically. From the area of computational fluid dynamics
\cite{wesseling2009principles}, it is known that a simple central differences
discretization scheme for Eq. (\ref{eq:almostdone3}) will be highly unstable due to
the introduction of strong \emph{numerical} dispersion near sharp edges or
discontinuities of the solution. A finite difference scheme that does not
generate such spurious oscillations is called monotonicity preserving
\cite{wesseling2009principles} and its usage is essential for the correct
interpretation of simulation results, especially when one tries to quantify
the amount of \emph{physical} dispersion present. Without getting too much
into detail, we present a second order linear finite difference discretization
scheme, possessing the monotonicity preserving property, for the model equation
\begin{equation}
\label{eq:model-equation}
\frac{\partial F}{\partial t}+ \frac{c}{n_{THz}}\frac{\partial F}{\partial
	x}=\eta(x,t)+kF,%
\end{equation}
where $k<0$ can incorporate arbitrary losses, and the rest of the terms can be renormalized in a straightforward way to obtain Eq. (\ref{eq:almostdone3}).

We take an equidistant spatio-temporal grid with grid size $\Delta x$ and time
step $\Delta t$ and set the values of the grid variables at spatial point
$x_{m}=m\Delta x$ and time $t_{n}=n\Delta t$ as $F(m,n)$ and $\eta(m,n)$ denotes the corresponding source term. The time stepping scheme we use is based on a second order upwind discretization, first introduced by Risken and Nummedal \cite{risken1968self}, and is given by
\begin{align}
\label{eq:numerics}
F(m,n+1) &  =F(m-1,n)+\Delta t\left[ \eta(m,n)+kF(m,n)\right]  \nonumber\\
&  +\frac{\Delta t^{2}}{2}\left\{  \left[  \frac{\partial \eta}{\partial
	t}\right]_{m}^{n}- \frac{c}{n_{THz}}\left[  \frac{\partial \eta}{\partial x}\right]
_{m}^{n}-2k\frac{c}{n_{THz}}\left[  \frac{\partial F}{\partial x}\right]_{m}%
^{n}+k\eta(m,n)+k^{2}F(m,n)\right\}  ,
\end{align}
where the \textbf{time step is chosen as $\Delta t=\Delta xn_{THz}/c$}, with $c/n_{THz}$
being the velocity of light in the medium. The evaluation of the time
derivative of $\eta(x,t)$ can be computed analytically from the density matrix
equations. The terms in rectangular brackets, $\left[
\partial \eta/\partial x\right]_{m}^{n}$ and $\left[  \partial F/\partial x\right]  _{m}^{n}$, can be computed via a simple first order forward
finite difference scheme (since we assume uni-directional propagation in the positive $x$-direction), which due to the prefactor of $\delta t^2$ will preserve the second order accuracy of the scheme.  
\bibliography{bib_resources.bib}
\end{document}

