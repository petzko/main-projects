%% ****** Start of file apsguide4-1.tex ****** %
%%
%%   This file is part of the APS files in the REVTeX 4.1 distribution.
%%   Version 4.1r of REVTeX, August 2010.
%%
%%   Copyright (c) 2009, 2010 The American Physical Society.
%%
%%   See the REVTeX 4.1 README file for restrictions and more information.
%%
\documentclass[reprint,secnumarabic,amssymb, nobibnotes, aip, prd]{revtex4-1}
%\usepackage{acrofont}%NOTE: Comment out this line for the release version!
\newcommand{\revtex}{REV\TeX\ }
\newcommand{\classoption}[1]{\texttt{#1}}
\newcommand{\macro}[1]{\texttt{\textbackslash#1}}
\newcommand{\m}[1]{\macro{#1}}
\newcommand{\env}[1]{\texttt{#1}}
\setlength{\textheight}{9.5in}


\usepackage{amsmath,amsfonts,amssymb}
\usepackage{graphicx}
\usepackage[colorlinks=true, allcolors=blue]{hyperref}


\usepackage{color}
\usepackage[latin9]{inputenc}
\usepackage{mathrsfs,amsmath}
\usepackage{graphicx}%
\usepackage{float}
\usepackage{amsfonts}%
\usepackage[titletoc]{appendix}
\usepackage{amssymb}
\usepackage{braket}
\usepackage{bm}

\newcommand{\mb}[1]{\bm{#1}}
\usepackage[T1]{fontenc}

\def\Nabla{\bm{\nabla}}
\def\bm{\mathbf}
\def\curl{\Nabla\times}
\def\div{\Nabla\cdot}
\def\lap{\Delta}
\def\vlap{\Delta}
\def\x{\hat{e}_{x}}
\def\y{\hat{e}_{y}}
\def\z{\hat{e}_{z}}
\def\p{\partial}
\def\h{\hat}
\def\h{\hat}
\def\tw{\tilde{\omega}}
\def\gm{\gamma}
\def\om{\omega}
\def\OM{\Omega}
\def\GM{\Gamma}
\def\dw{\delta\omega}
\def\dth{\Delta\theta}
\def\dk{\delta k}
\def\Hdth{\frac{\dth}{2}} %half Delta Theta
\def\P{\hat{\pi}_+}
\def\M{\hat{\pi}_-}
\newcommand{\vspacec}{\vspace{-0.3cm}}
%\usepackage[font=small]{caption}
\newcommand{\includegraphicsXL}[1]{\includegraphics[width = 0.40\textwidth]{#1}}
%\captionsetup{width=.45\textwidth}




\bibliographystyle{ieeetr}
\begin{document}

\title{Temporal hole burning in lasers with an upper state doublet}%

\author{Petar Tzenov}%
\email{petar.tzenov@tum.de}
\affiliation{Institute for Nanoelectronics, Technical University of Munich, D-80333 Munich, Germany}
\author{David Burghoff}
\affiliation{Department of Electrical Engineering
	and Computer Science, Research Laboratory of Electronics, Massachusetts
	Institute of Technology, Cambridge, Massachusetts 02139, USA}
\author{Qing Hu}
\affiliation{Department of Electrical Engineering
	and Computer Science, Research Laboratory of Electronics, Massachusetts
	Institute of Technology, Cambridge, Massachusetts 02139, USA}
\author{Christian Jirauschek}
\affiliation{Institute for Nanoelectronics, Technical University of Munich, D-80333 Munich, Germany}

\begin{abstract}
We present a theoretical explanation for a novel phenomenon, termed temporal hole burning (THB), first experimentally observed in terahertz quantum cascade laser (THz QCL) frequency combs. We trace the origins of THB to resonant tunneling induced spectral splitting of the injector and the upper laser level of these devices into a doublet of anticrossed states, both of which are radiatively coupled to the ground state. In the time-domain, the existence of a pair of upper laser levels produces two co-propagating pulses which do not coexist in time, but transiently switch on and off as to maintain an almost constant instantaneous intensity. With the aid of a simple three level density matrix formalism, we show that THB can be interpreted as a two-photon process where the distinct lasing modes exchange a photon accompanied by an energy-conserving electron redistribution in the system. Fundamentally, such a dynamics reminds us of stimulated Raman scattering, however with an inverted energy configuration as the electron exchange occurs via the ground state, rather than the upper levels.  
\end{abstract}
\maketitle

Resonant tunneling (RT) is a coherent quantum phenomenon where electrons can tunnel through a pair of potential barriers into an available state on the other end of this energetic obstacle \cite{davies1997physics}. This process is most efficient, i.e. the transmission is maximized, when both in-coming and out-going wave functions are perfectly aligned in energy, with the tunneling current taking the shape of a Lorentzian centred around the resonance condition. The tunneling effect is well understood and widely used in various quantum devices such as the Zener or resonant-tunneling diodes \cite{tsu1973tunneling}, as well as quantum well heterostructures \cite{kazarino1971possibility}. 


Terahertz quantum cascade lasers (THz QCLs) routinely exploit RT for selective electron injection into the upper laser state, and correspondingly efficient electron extraction from the depopulation level \cite{sirtori1998resonant,williams2007terahertz}. Several theoretical investigations have been made on the impact of different system parameters, such as detuning energy, coupling strength and dephasing, on the coherence of tunneling transport, but all of those discussions have neglected transient effects and tackled the problem in steady state \cite{callebaut2005importance,kumar2009coherence,dupont2010simplified}. For example, it was quickly understood \cite{dupont2010simplified} that in frequency domain, for devices with sufficiently coherent optical transition or sufficiently strong anticrossing energy, QCLs employing resonant tunneling coupling between the injector and the upper laser level should produce a split optical spectrum, symmetrically distributed around the upper-lower laser level transition frequency. Due to the inherent difficulties in designing a sufficiently coherent broadband QC laser, such a spectral splitting, the origins of which lie in the RT coupling-induced anticrossing,  was experimentally observed only in 2014 \cite{burghoff2014terahertz}. In this publication the authors presented a THz frequency comb, consisting of more than 70 equidistantly spaced modes, with a spectrum comprised of a low and a high frequency lobe centred around 3.4 THz and 3.8 THz, respectively, separated without significant cross-talk (Fig.~\ref{fig:thb_experiment}a). Subsequent investigation of the time-domain dynamics of this device \cite{burghoff2015evaluating} revealed a very intricate pulse switching behaviour, named "temporal hole burning" by the authors, which left for a concise theoretical explanation and will be the main focus of this letter. 
\begin{figure}[h!]
	\begin{center}
		\includegraphicsXL{IMGS/exper_0p9A_small.eps}
		\caption{Experimental data from the device in Ref.~\onlinecite{burghoff2014terahertz} in frequency and time domain, when the laser was injected with 0.9 A current. \textbf{a} Optical spectrum, and \textbf{b} the averaged out instantaneous intensity, $I(t)$, obtained via the method in Ref.~\onlinecite{burghoff2015evaluating} over a smoothing window of 10 ps. The high frequency lobe in \textbf{a} corresponds to the pulse in \textbf{b} with longer duration, coloured in dark blue, whereas the low frequency lobe to the shorter pulse, coloured in red.} \label{fig:thb_experiment}
	\end{center}	
\end{figure}

\vspacec
Figure~\ref{fig:thb_experiment} illustrates experimental data \cite{burghoff2015evaluating}, demonstrating the temporal hole burning effect in both time and frequency domains. In the intensity spectrum of Fig.~\ref{fig:thb_experiment}a we can observe spectral hole burning as the optical power is distributed into two lobes, as mentioned above and in good agreement with theoretical predictions. On the other hand, in the time domain, i.e. Fig.~\ref{fig:thb_experiment}b, we see how the blue and the red pulses do not coexist in time, but instead periodically "burn a hole" into each other's temporal trace. It is interesting to note that both signals combine as to produce an almost constant instantaneous intensity, a feature of QCL combs produced by self frequency modulation\cite{khurgin2014coherent}. In our previous publication \cite{petz2016} we developed a density matrix model, coupled to Maxwell's equations, for time-domain simulations of THz QC lasers. When we modelled the comb in Ref.~\onlinecite{burghoff2014terahertz}, our results reproduced the spectral splitting and more interestingly the pulse switching behaviour, as illustrated in Fig.~\ref{fig:thb_simulation}, which lead us to think that THB is intrinsically included in our model. In the current paper we set out to explain this effect analytically with the aid of a substantial simplification of our previous work, which however still correctly captures the essential physics. In contrast to Ref.~\onlinecite{petz2016}, here we move away from the tight binding (TB) basis, Fig.~\ref{fig:basisnew}a, traditionally used to model resonant tunneling phenomena \cite{callebaut2005importance}, and consider the interaction of a unidirectional electric field with a three level system in the delocalized picture, Fig.~\ref{fig:basisnew}b. In all subsequent derivations we also omit the inclusion of incoherent intersubband scattering or dephasing processes, as we do not consider these effects as critical for the understanding of temporal hole burning. 
\begin{figure}[h!]
	\begin{center}
		\includegraphicsXL{IMGS/sim_10p8_small.eps}
		\caption{Simulation results from the our model in Ref.~\onlinecite{petz2016}, where all parameters are as in the reference with the exception of the pure dephasing time, which was set to 2 ps in order to produce more pronounced spectral splitting. \textbf{a} Simulated optical spectrum and \textbf{b} the computed instantaneous intensity $I(t)$ of the high and low frequency lobes. The colour-coding is identical to Fig.~\ref{fig:thb_experiment}.}\label{fig:thb_simulation}
	\end{center}	
\end{figure}

\vspacec
The energy splitting, or spectral hole, induced by the resonant tunneling coupling, can be most easily understood by comparing the tight binding with the extended basis pictures, as illustrated in Fig.~\ref{fig:basisnew}. In the TB basis, Fig.~\ref{fig:basisnew}a, the active region states $\ket{l}$ and $\ket{g}$ (for "left" and "ground", respectively) are spatially localized on the left side of the tunneling barrier, whereas the injector state $\ket{r}$ is confined on the right side. We set $\hbar\omega_0$ to denote the $\ket{l} \leftrightarrow \ket{g}$ transition energy and $\hbar\epsilon$ the detuning between the states $\ket{r}$ and $\ket{l}$. When $\hbar \epsilon \ll 1 $ meV, the close energetic alignment of the left and the right levels induces a strong wave function coupling, modelled via the energy interaction term $\hbar\Omega_{lr}$. This leads to a regime of coherent electron transport across the barrier via Rabi oscillations \cite{callebaut2005importance}. An equivalent, and probably more convenient for our purposes, interpretation is the delocalized, or dressed states, picture, illustrated in Fig.~\ref{fig:basisnew}b. Since the total system's Hamiltonian is non-diagonal in the TB basis, the inclusion of the coupling term $\hbar\Omega_{lr}$ produces a splitting of the near-resonant states $\ket{l}$ and $\ket{r}$ into a new doublet, the symmetric $\ket{-}$ and anti-symmetric $\ket{+}$ levels, spatially extended across the tunneling barrier. Essentially this results in a pair of upper levels, from both of which radiative transition to the ground state is possible. It is within this picture that the temporal hole burning effect is susceptible to a very intuitive explanation. 
\begin{figure}[h!]
	\begin{center}
		\includegraphicsXL{IMGS/basisnew.eps}
		\caption{3-level system, employing RT for current injection, represented in the \textbf{a} tight binding and \textbf{b} delocalized basis. } \label{fig:basisnew}
	\end{center}	
\end{figure}


In operator formalism, neglecting energy dissipation due to interaction with the environment, the atomic Hamiltonian of the unperturbed system in Fig.~\ref{fig:basisnew}b can be written down as 
\begin{align}
\label{eq:semiclassical_H}
\hat{H} &= \hbar \omega_+\P^{\dagger}\P+\hbar\omega_-\M^{\dagger}\M, 
\end{align}
where $\hat{\pi}_{\pm} = \ket{g}\bra{\pm}$ and $\hat{\pi}_{\pm}^{\dagger} = \ket{\pm}\bra{g}$ are the lowering and raising operators, respectively, modelling the possible transitions from the anti-symmetric and symmetric dressed states to the ground level, $\ket{g}$. In Eq.~(\ref{eq:semiclassical_H}) we have set the eigenenergies of the anticrossed states  $E_{\pm} = \hbar\omega_{\pm} = \hbar\omega_0\pm \hbar[\epsilon^2+4\Omega_{lr}]^{1/2}/2$, and that of the ground level to zero, i.e. $E_g=0$. To include the light-matter interaction dynamics, we couple the atomic system to a classical electric field propagating in the $x$-direction, only the $z$-component of which, i.e. $E_z(x)$, interacts with the conduction band electrons. We further decompose the field $E_z(x) = \Re\left\{f(x,t)e^{i(k_0x-\omega_0 t)}\right\}$ as the product of a slowly-varying envelope, $f(x,t)$, and a quickly varying carrier wave, oscillating with the $g\leftrightarrow l$ transition frequency, $\omega_0$. The corresponding wave number is $k_0=n_0\omega_0/c$, where $n_0$ denotes the refractive index and $c$ the velocity of light in vacuum. Lastly, we assume the electric dipole and the rotating wave approximations and write down the resulting semi-classical Hamiltonian as follows
\begin{align}
\label{eq:semiclassical_H}
\hat{H} &= \hbar (\omega_+-\omega_0)\P^{\dagger}\P+\hbar(\omega_--\omega_0)\M^{\dagger}\M \nonumber \\ 
&+\frac{d_{+}}{2}(f\P^{\dagger}+f^*\P)+\frac{d_{-}}{2}(f\M^{\dagger}+f^*\M),
\end{align}
where $d_{\pm} = q_0\bra{\pm}\hat{z}\ket{g} $ are the transition dipoles and $q_0$ denotes the elementary charge. 

The density matrix of this 3-level system can be defined as 
\begin{align}
\rho = \begin{bmatrix}
\rho_{++} & \rho_{+-} & \eta_{+g} \\ 
\rho_{-+} & \rho_{--} & \eta_{-g} \\
\eta_{g+} & \eta_{g-} & \rho_{gg}
\end{bmatrix},
\end{align}
where $\rho_{ii}$ denote the population densities and $\eta_{\pm g} = \rho_{\pm g}e^{i\omega_0t}$ the slowly varying coherences. The time evolution of $\rho$ is governed by the von Neumann equation, which for the $\rho_{++}$, $\rho_{--}$, and $\eta_{\pm g}$ matrix elements gives  
\begin{subequations}
	\label{eq:4eqns}
	\begin{align}
	\frac{d \rho_{++}}{dt} &= i\frac{d_{+}}{2\hbar}(f^*\eta_{+g}-f\eta_{+g}^*), \\
	\frac{d \rho_{--}}{dt} &= i\frac{d_{-}}{2\hbar}(f^*\eta_{-g}-f\eta_{-g}^*), \\
	\frac{d \eta_{+g}}{dt} &= -i\Omega_{lr}\eta_{+g}+i\frac{d_{+}}{2\hbar}f(\rho_{++}-\rho_{gg})+i\frac{d_{-}}{2\hbar}f\rho_{+-}, \label{eq:eta+a}\\
	\frac{d \eta_{-g}}{dt} &= i\Omega_{lr}\eta_{-g}+i\frac{d_{-}}{2\hbar}f(\rho_{--}-\rho_{gg})+i\frac{d_{+}}{2\hbar}f\rho_{-+}, \label{eq:eta-a}
	\end{align}
\end{subequations}
where we have assumed that $\epsilon = 0$. 

The origin of the temporal hole burning effect is revealed by assuming that $f = f^*$, which is reasonable under resonance, since then the even order dispersion can be considered negligible\cite{khurgin2005optical} (see supplementary material for detailed justification), as well as taking $d_{+}\approx d_{-} = d$, setting  $\eta = \eta_{+g}+(\eta_{-g})^*$ and deriving the equation of motion of this quantity. Direct substitution from Eq.~(\ref{eq:4eqns}) gives us
\begin{equation}
\label{eq:coherence_quasi2lvl}
\frac{d \eta}{dt} = -i\Omega_{lr} \eta + i\beta w, 
\end{equation}
where $w = \rho_{++}-\rho_{--}$ denotes the inversion between the delocalized states and $\beta =d f/2\hbar$ is the Rabi frequency. This inversion evolves according to 
\begin{align}
\label{eq:inversion_quasi2lvl}
\frac{d w }{dt}	&=  i\beta(\eta-\eta^*)
\end{align}
from where we see that the variables $\eta$ and $w$ form a "quasi-particle", the time-evolution of which is completely decoupled from the rest of the system. However, one needs to notice that Eqs.~(\ref{eq:coherence_quasi2lvl}) and (\ref{eq:inversion_quasi2lvl}) differ from the usual Bloch equations \cite{boyd2003nonlinear}, due to the absence of a factor of 2 in the right hand side of Eq.~(\ref{eq:inversion_quasi2lvl}), which, as will be shown below, constrains the evolution of the system on the surface of an ellipsoid rather than a sphere in Bloch space.

From a more fundamental point of view, we can look at the temporal hole burning dynamics as a two-photon process where the states $\ket{+}$ and $\ket{-}$ exchange an electron via the ground state $\ket{g}$. This naturally follows from the interpretation of $\eta = \eta_{+g}+\eta_{g-}$ as expectation value of the time-dependent operator \cite{loudon2000quantum}
\begin{align}
\hat\sigma(t) = \P(t) +\M^{\dagger}(t),
\end{align}
where the terms $\hat\pi_{\pm}(t)$ are the slowly varying Heisenberg-picture analogues of the raising and lowering operators introduced before, satisfying $\P(0) =\ket{g}\bra{+}$ and $\M^{\dagger}(0)=\ket{-}\bra{g}$. We can thus see that $\hat\sigma$ will lower an electron from the anti-symmetric state, $\ket{+}$, to the ground level and raise an electron from $\ket{g}$ to the symmetric state, $\ket{-}$. This process is accompanied by an energy conserving exchange of a photon in the $\omega_{-}$ mode for a photon in the $\omega_{+}$ mode, as illustrated in Fig.~\ref{fig:two_photon_process}a, and continues until the anti-symmetric state is sufficiently depleted so that the conjugate effect could take over. Figure~\ref{fig:two_photon_process}b illustrates this alternative process, namely the one corresponding to the action of the conjugate transpose operator $\hat{\sigma}^{\dagger}$, which swaps a $\omega_+$ photon for a $\omega_-$ photon. In a way, these two-photon processes are similar to the familiar Stokes, Fig.~\ref{fig:two_photon_process}c, and anti-Stokes, Fig.~\ref{fig:two_photon_process}d, Raman scattering, however in an flipped energy configuration as the electron exchange does not occur in the upper levels, but rather in the ground state \cite{butcher1991elements}. 
\begin{figure}[h!]
	\begin{center}
		\includegraphicsXL{IMGS/two_photon_process.eps}
		\caption{Schematic illustration of the two-photon processes, responsible for temporal hole burning, \textbf{a} and \textbf{b}, and the familiar Stokes, \textbf{c}, and anti-Stokes, \textbf{d}, Raman scattering.} \label{fig:two_photon_process}
	\end{center}	
\end{figure}

\vspacec
Semi-classically, we can also see that Eqs.~(\ref{eq:coherence_quasi2lvl}) and (\ref{eq:inversion_quasi2lvl}) would lead to a pulse switching behaviour, if we examine the field envelope's time-evolution equation. From Maxwell's equations, in the absence of free electric charges and assuming weak inhomogeneities, the slowly varying envelope approximation yields the propagation equation \cite{jirauschek2014modeling}
\begin{align}
\label{eq:propagation01}
\frac{n_0}{c} \p_t f + \p_x f = -i\frac{N \Gamma d k_0}{\epsilon_0 n_0^2}(\eta_{+g}+\eta_{-g})
\end{align}
for the envelope $f(x,t)$, where $\epsilon_0$ is the permittivity of free space, $N$ is the average carrier density per unit volume and $\Gamma$ is the overlap factor. Now, if we decompose $\eta_{+g}$ and $\eta_{-g}$ into their real and imaginary parts, i.e. $\eta_{+g} = u_{+}+iv_{+}$ and $\eta_{-g} = u_{-}+iv_{-}$, and substitute into Eq.~($\ref{eq:propagation01}$), we see that the real part of the total polarization, i.e. $u_++u_-$, models changes in the envelope's phase, i.e. dispersion, whereas the imaginary part, $v_++v_-$, captures the gain or losses due to the atomic resonances. On the other hand, Eq.~(\ref{eq:coherence_quasi2lvl}) evolves the \emph{difference} of $v_+$ and $v_-$, i.e. $\Im\{\eta\}=v_+-v_-$, and not their sum. This means that, for example, when $\Im\{\eta\}$ is a large  positive number the $\ket{+}\leftrightarrow \ket{g}$ transition will experience gain, and the $\ket{-}\leftrightarrow\ket{g}$ one loss and when $\Im\{\eta\}$ is negative - vice versa. Note that this is namely the classical analogue of the photon-exchange picture discussed above.


%To enhance this interpretation let us write Eq. (\ref{eq:coherence_quasi2lvl}) and Eq. (\ref{eq:inversion_quasi2lvl}) into mathematically equivalent matrix form as follows
%\begin{align}
%\label{eq:QB_matrix}
%\dot{\rho}_{QB} = -i[H_{QB},\rho_{QB}] +\beta \begin{bmatrix}
%\Im\{\eta\} & 0 \\ 0 & -\Im\{\eta\}
%\end{bmatrix},
%\end{align}
%where the quasi-Bloch Hamiltonian and statistical operator take the form
%\begin{align}
%\label{eq:QB_H_RHO}
%H_{QB}= \begin{bmatrix}
%\Omega & \beta \\ \beta & 0
%\end{bmatrix}, \quad \rho_{QB}= \begin{bmatrix}
%\rho_{++} & \eta \\ \eta^* & \rho_{--}
%\end{bmatrix}.
%\end{align}


Lastly, we can elaborate a little further and make the ansatz $u=\Re\{\eta\}$, $v =\Im\{\eta\}$ and as before $w = \rho_{++}-\rho_{--}$. This transforms Eqs.~(\ref{eq:coherence_quasi2lvl}) and (\ref{eq:inversion_quasi2lvl}) into the following system 
\begin{subequations}
	\label{eq:quasi-Blocheqn}
	\begin{align}
		\dot{u} &= \Omega_{lr} v , \\
		\dot{v} &= -\Omega_{lr} u +\beta w , \\
		\dot{w} &= -2\beta v.
	\end{align}
\end{subequations}

 Notice that Eqs.~(\ref{eq:quasi-Blocheqn}) above are \emph{not} exactly the familiar Bloch equations and therefore the solutions do \emph{not} lie on the Bloch sphere. In fact, we see that $d(u^2+v^2+w^2)/dt = -2\beta vw \neq 0$, i.e. the radius is actually not a conserved quantity. One can easily show that the Bloch-space variables, $u,v $ and $w$, instead satisfy the equality $d(u^2+v^2+(w/\sqrt{2})^2)/dt = 0$, which describes the surface of an ellipsoid with two equal principal axes $a=b$, i.e. those in the $u-v$ plane, and a third axis $c = \sqrt{2}a$ parallel to $w$, which tells us that, irrespective of the driving field $\beta(t)$, the time evolution will proceed on the surface of an ellipsoid in Bloch space. 
\begin{figure}[h!]
	\begin{center}
		\includegraphics[width = 0.45\textwidth]{IMGS/quasi-bloch-resonance-new.eps}
		\caption{\textbf{a} The Bloch variables $u,v$ and $w$ evolved for 120 picoseconds with Eq.~(\ref{eq:quasi-Blocheqn}), when $\omega_f = \Omega_{lr}$ and $\beta_0 = 0.1\times\Omega_{lr}$, with $\Omega_{lr} = 290\times 2\pi$ GHz. \textbf{b} Bloch space representation of \textbf{a}.} \label{fig:quasi-bloch-resonance}
	\end{center}	
\end{figure}

\vspacec
We solve these quasi-Bloch equations, i.e. Eqs.~(\ref{eq:quasi-Blocheqn}), numerically, initially setting $u(t=0)=0$, $v(t=0) = 0$ and $w(t=0) = 0.2$ and assuming $\beta(t) = \beta_0 \cos\omega_f t$. The latter is indeed a reasonable assumption as the field envelope $f(t)$ will not be a DC field, but rather contain components oscillating around the frequency $\omega_f \approx \Omega_{lr}$, which directly follows from the tunneling-induced energy splitting. Our results, when we set $\beta_0 = 0.1\Omega_{lr}$ and $\omega_f = \Omega_{lr}$, are plotted in Fig.~\ref{fig:quasi-bloch-resonance}. These simulations reveal that the population inversion, $w$, completes full swings between -0.2 and 0.2, whereas in the same time $|u|$ and $|v|$ are upper bounded by $0.2/\sqrt{2}$, Fig.~\ref{fig:quasi-bloch-resonance}a. This confirms the deduction that the coherent time-evolution of this system will lie on the surface of an ellipsoid with principal axes $a=b=c/\sqrt{2}$ in the $u$,$v$ and $w$ direction, respectively, as it is evident from Fig.~\ref{fig:quasi-bloch-resonance}b. Additionally, we can observe that the system dynamics is governed by two time scales: a fast one, having a period close to $2\pi/\Omega_{lr}$, which follows the envelope frequency, and another, slower time-scale originating from the field-induced Rabi-oscillations, the period of which depends on the magnitude of $\beta_0$.  

In summary, we have presented experimental evidence and theoretical explanation for a novel time-domain effect, named temporal hole burning, which in addition to spectral and spatial hole burning, completes the classification of hole burning phenomena encountered in  literature. THB was first observed in a terahertz QC laser-based frequency comb with a strong injector anticrossing\cite{burghoff2015evaluating}, however in principle it could be detected in any sufficiently coherent and broadband laser with an upper laser level doublet. Ideally, the signal produced by such a device would consists of two pulses propagating in a way as to "burn" a hole onto each other's instantaneous intensities. Under the presented simplifying assumptions, temporal hole burning can be seen as a two-photon process similar to the familiar stimulated Raman scattering, however with inverted energy level configuration. We note that THB persists also when those conditions are slightly violated, as already shown by experiment\cite{burghoff2015evaluating} and simulation\cite{petz2016}, however in those cases the system of equations does not simplify sufficiently to clearly reveal the true nature of this mechanism. 

This work was supported by the German Research Foundation (DFG) within the Heisenberg program (JI 115/4-1) and under DFG Grant No. JI 115/9-1.
\bibliography{D:/docs/MAIN-PROJECTS/PAPERS/literature/bib_resources.bib}
\end{document}

