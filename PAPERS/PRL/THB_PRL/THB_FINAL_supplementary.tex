%% ****** Start of file apsguide4-1.tex ****** %
%%
%%   This file is part of the APS files in the REVTeX 4.1 distribution.
%%   Version 4.1r of REVTeX, August 2010.
%%
%%   Copyright (c) 2009, 2010 The American Physical Society.
%%
%%   See the REVTeX 4.1 README file for restrictions and more information.
%%
\documentclass[preprint,secnumarabic,amssymb, nobibnotes, aip, prd]{revtex4-1}
%\usepackage{acrofont}%NOTE: Comment out this line for the release version!
\newcommand{\revtex}{REV\TeX\ }
\newcommand{\classoption}[1]{\texttt{#1}}
\newcommand{\macro}[1]{\texttt{\textbackslash#1}}
\newcommand{\m}[1]{\macro{#1}}
\newcommand{\env}[1]{\texttt{#1}}
\setlength{\textheight}{9.5in}


\usepackage{amsmath,amsfonts,amssymb}
\usepackage{graphicx}
\usepackage[colorlinks=true, allcolors=blue]{hyperref}


\usepackage{color}
\usepackage[latin9]{inputenc}
\usepackage{mathrsfs,amsmath}
\usepackage{graphicx}%
\usepackage{float}
\usepackage{amsfonts}%
\usepackage[titletoc]{appendix}
\usepackage{amssymb}
\usepackage{braket}
\usepackage{bm}
\usepackage{xr}
\externaldocument[MAIN-]{THB_FINAL}
\newcommand{\mb}[1]{\bm{#1}}
\usepackage[T1]{fontenc}

\def\Nabla{\bm{\nabla}}
\def\bm{\mathbf}
\def\curl{\Nabla\times}
\def\div{\Nabla\cdot}
\def\lap{\Delta}
\def\vlap{\Delta}
\def\x{\hat{e}_{x}}
\def\y{\hat{e}_{y}}
\def\z{\hat{e}_{z}}
\def\p{\partial}
\def\h{\hat}
\def\h{\hat}
\def\tw{\tilde{\omega}}
\def\gm{\gamma}
\def\om{\omega}
\def\OM{\Omega}
\def\GM{\Gamma}
\def\dw{\delta\omega}
\def\dth{\Delta\theta}
\def\dk{\delta k}
\def\Hdth{\frac{\dth}{2}} %half Delta Theta
\def\P{\hat{\pi}_+}
\def\M{\hat{\pi}_-}
\newcommand{\vspacec}{\vspace{-0.3cm}}
%\usepackage[font=small]{caption}
\newcommand{\includegraphicsXL}[1]{\includegraphics[width = 0.75\textwidth]{#1}}
%\captionsetup{width=.45\textwidth}


\bibliographystyle{ieeetr}
\begin{document}
	
	\title{Temporal hole burning in lasers with an upper state doublet \\ (Supplementary material)}%

	\author{Petar Tzenov}%
	\email{petar.tzenov@tum.de}
	\affiliation{Institute for Nanoelectronics, Technical University of Munich, D-80333 Munich, Germany}
	\author{David Burghoff}
	\affiliation{Department of Electrical Engineering
		and Computer Science, Research Laboratory of Electronics, Massachusetts
		Institute of Technology, Cambridge, Massachusetts 02139, USA}
	\author{Qing Hu}
	\affiliation{Department of Electrical Engineering
		and Computer Science, Research Laboratory of Electronics, Massachusetts
		Institute of Technology, Cambridge, Massachusetts 02139, USA}
	\author{Christian Jirauschek}
	\affiliation{Institute for Nanoelectronics, Technical University of Munich, D-80333 Munich, Germany}
	\maketitle
\section{Detailed derivation of the temporal hole burning model}

As already suggested in the main body of the paper, the coupling of the injector and the upper laser states leads to their splitting into a doublet of anticrossed levels, spatially extending over the tunneling barrier. The original "tight-binding" states, i.e. $\ket{g}$, $\ket{l}$ and $\ket{r}$, are related to the delocalized basis states, i.e. $\ket{g}$, $\ket{-}$ and $\ket{+}$,  via the unitary transform 
\begin{align}
\ket{+} &= \cos\theta\ket{r}-\sin\theta\ket{l}, \\
\ket{-} &= \sin\theta\ket{r}+\cos\theta\ket{l}, \\
\ket{g} &=\ket{g},
\end{align}
where the expansion coefficients are obtained from $\tan\theta = -2\Omega_{lr}/[\epsilon+(\epsilon^2+4\Omega_{lr}^2)^{1/2}]$. Here $\epsilon$ is the resonant tunneling detuning angular frequency and $\Omega_{lr}$ the anticrossing strength. This will lead to a transfer of the transition dipole moment $d_{lg}=q_0\bra{l}\hat{z}\ket{g}$ onto the pair $d_{+}$ and $d_{-}$, defined as in the main body of the paper, which approximately satisfy the relation $d_{lg}^2=d_{+}^2+d_{-}^2$. In the $\{\ket{+},\ket{-},\ket{g}\}$ basis, the Hamiltonian takes the form
\begin{align}
H = \begin{bmatrix}
\hbar\omega_+ & 0 & d_+E_z \\
0 & \hbar\omega_- & d_-E_z \\
d_+E_z & d_-E_z & 0
\end{bmatrix},
\end{align}
which, when plugged into the von Neumann equation, leads to the system
\begin{subequations}
\label{eq:vonNeu_nonapprox}
\begin{align}
\frac{d \rho_{++}}{dt} &= i\frac{d_+E_z}{\hbar}(\rho_{+g}-\rho_{+g}^*), \\
\frac{d \rho_{--}}{dt} &= i\frac{d_-E_z}{\hbar}(\rho_{-g}-\rho_{-g}^*), \\
\frac{d \rho_{gg}}{dt} &= -i\frac{d_+E_z}{\hbar}(\rho_{+g}-\rho_{+g}^*)-i\frac{d_-E_z}{\hbar}(\rho_{-g}-\rho_{-g}^*), \\
\frac{d \rho_{+-}}{dt} &= -i(\omega_+-\omega_-)\rho_{+-}-i\frac{d_{+}E_z}{\hbar}\rho_{g-}+i\frac{d_{-}E_z}{\hbar}\rho_{+g},\\
\frac{d \rho_{+g}}{dt} &= -i\omega_+\rho_{+g}+i\frac{d_{+}E_z}{\hbar}(\rho_{++}-\rho_{gg})+i\frac{d_{-}E_z}{\hbar}\rho_{+-}, \\
\frac{d \rho_{-g}}{dt} &= -i\omega_-\rho_{-g}+i\frac{d_{-}E_z}{\hbar}(\rho_{--}-\rho_{gg})+i\frac{d_{+}E_z}{\hbar}\rho_{-+}. 
\end{align}
\end{subequations}
Now, to employ the rotating wave approximation \cite{weiner2011ultrafast}, we make the standard ansatz 
\begin{align}
E_z(x,t) &= \Re\{f(x,t)e^{i(k_0x-\omega_0t)}\}, \\
\rho_{+g}(x,t) &= \eta_{+g}(x,t)e^{i(k_0x-\omega_0t)}, \\
\rho_{-g}(x,t) &= \eta_{-g}(x,t)e^{i(k_0x-\omega_0t)},
\end{align}
where $\omega_0$ is the $\ket{l}\leftrightarrow\ket{g}$ transition frequency and $k_0$ the corresponding wave number.  We finally plug this ansatz into Eqs.~(\ref{eq:vonNeu_nonapprox}) and drop terms oscillating with $e^{\pm 2i\omega_0t}$. From this procedure we obtain the time evolution of $\rho^{\text{RWA}}$ as
\begin{subequations}
\label{eq:4eqns}
\begin{align}
\frac{d \rho_{++}}{dt} &= i\frac{d_{+}}{2\hbar}(f^*\eta_{+g}-f\eta_{+g}^*), \\
\frac{d \rho_{--}}{dt} &= i\frac{d_{-}}{2\hbar}(f^*\eta_{-g}-f\eta_{-g}^*), \\
\frac{d \rho_{gg}}{dt} &= - i\frac{d_{+}}{2\hbar}(f^*\eta_{+g}-f\eta_{+g}^*)-i\frac{d_{-}}{2\hbar}(f^*\eta_{-g}-f\eta_{-g}^*), \\
\frac{d \rho_{+-}}{dt} &= -i(\omega_+-\omega_-)\rho_{+-}-i\frac{d_{+}f}{2\hbar}\eta_{g-}+i\frac{d_{-}f^*}{2\hbar}\eta_{+g},\\
\frac{d \eta_{+g}}{dt} &= -i(\omega_+-\omega_{0})\eta_{+g}+i\frac{d_{+}}{2\hbar}f(\rho_{++}-\rho_{gg})+i\frac{d_{-}}{2\hbar}f\rho_{+-}, \label{eq:eta+g}\\
\frac{d \eta_{-g}}{dt} &= -i(\omega_--\omega_{0})\eta_{-g}+i\frac{d_{-}}{2\hbar}f(\rho_{--}-\rho_{gg})+i\frac{d_{+}}{2\hbar}f\rho_{-+}. \label{eq:eta-g}. 
\end{align}
\end{subequations}
We can cast Eqs. (\ref{eq:4eqns}) into their final form, i.e. Eqs. (\ref{MAIN-eq:4eqns}) from the main body of the article, via assuming that when resonance detuning is small, i.e. $\epsilon \approx  0$ , then the anticrossing splitting yields the energy difference $\omega_+-\omega_0 \approx \Omega_{lr}$, $\omega_--\omega_0\approx - \Omega_{lr}$ as well as $\omega_+-\omega_- \approx 2\Omega_{lr}$. Next we assume that the dipoles have the same algebraic value, i.e. $d_+ = d_- =d$, and that the optical field's Rabi frequency, i.e. $\beta(x,t) = df(x,t)/2\hbar $ satisfies the relation $\Re\{\beta\} \gg \Im\{\beta\}$, i.e. it evolves primarily in the domain of real numbers. 

Under these conditions we can decouple the system of ODEs in Eq. (\ref{eq:4eqns}) by setting $\eta = \eta_{+g}+\eta_{-g}^*$, $\sigma = \eta_{+g}-\eta_{-g}^*$, $w = \rho_{++}-\rho_{--}$ and deriving the time-evolution of these quantities.

After some simple algebraic operations, neglecting terms $\propto \Im \{\beta\}$, we obtain
\begin{subequations}
	\label{eq:rhogg-system}
	\begin{align}
	\frac{d \rho_{gg}}{dt} &= - i\Re\{\beta\}(\sigma-\sigma^*), \label{eq:rhoggprime} \\
	\frac{d \rho_{+-}}{dt} &= -2i\Omega_{lr}\rho_{+-}+i\Re\{\beta\}\sigma,\\
	\frac{d \sigma}{dt} &= -i\Omega_{lr}\sigma+i\Re\{\beta\}(1-3\rho_{gg})+2i\Re\{\beta\}\rho_{+-},
	\end{align}
\end{subequations}
for $\rho_{gg}$, $\sigma$ and $\rho_{+-}$, as well as 
\begin{subequations}
\label{eq:w-system}
\begin{align}
\frac{d w}{dt} &= i\Re\{\beta\}(\eta-\eta^*), \\
\frac{d \eta}{dt} &= -i\Omega_{lr}\eta+i\Re\{\beta\} w, \label{eq:eta}
\end{align}
\end{subequations}
for $w$ and $\eta$. 
On the other hand, the Rabi frequency obeys the slowly varying propagation equation \cite{jirauschek2014modeling}
\begin{align}
\label{eq:propagation-eq}
\frac{n_0}{c}\frac{\p \beta}{\p t} + \frac{\p \beta}{\p x} &= - i\frac{N\Gamma d^2 k_0}{2\epsilon_0n_0^2\hbar}(\eta_{+g}+\eta_{-g}) \nonumber \\
		&=  -i\frac{g}{2}\Re{\eta}+\frac{g}{2}\Im{\sigma},
\end{align}
where $g =N\Gamma d^2 k_0/\epsilon_0n_0^2\hbar $ is the peak value of the optical gain. 

To completely decouple Eqs. (\ref{eq:rhogg-system}) from Eqs. (\ref{eq:w-system}) and (\ref{eq:propagation-eq}), we simply have to find a solution of Eqs. \ref{eq:rhogg-system}, for which $\Im\{\sigma\} =0$. If such a solution existed, this would have two very significant consequences. First, setting $\Im\{\sigma\} =0$ in Eq. (\ref{eq:propagation-eq}) would mean that the real part of the Rabi frequency would propagate without interacting with the medium, i.e. there seems to be a . And second, the population of the lower electron level, $\rho_{gg}$, would remain constant, which is evident from Eq. (\ref{eq:rhoggprime}). This observation strongly enhances our hypothesis of the onset of a two-photon process, suggested in the main body of this paper.

The last remaining questions left for us to address are whether it is reasonable to assume that the $\Re\{\beta\} \gg \Im\{\beta\}$, whether there is a solution of Eqs. (\ref{eq:rhogg-system}), for which $\Im\{\sigma\} = 0$ and how stable it is, and lastly what if the dipole moments have same magnitude but different algebraic sign, i.e. $d_+ = -d_-$. Each of these issues will be addressed briefly in the following parts of the supplementary material.  



\section{Validity of the temporal hole burning assumptions}
A few comments on the assumptions made in the derivation of the quasi-Bloch equations are in order. 


TODO:

Mention that at least one solution exists for the case when $\rho_{gg} = 1/3$ and $\sigma = \rho_{\pm} = 0$. Then Eqs. (\ref{eq:rhogg-system}) are identically satisfied and thus the systems have been decoupled. In the more general case, impose $\Im\{\sigma\}=0$ into Eq. (\ref{eq:rhogg-system}), amounts to solving a liner differential-algebraic equation with time-varying coefficients of the form 
\begin{align}
\dot{y} &= M(t)y \label{eq:diffequation} \\
0 &= g(y) \label{eq:algebraicequation},
\end{align}
where $y = [\Re\{\rho_{\pm}\},\Im\{\rho_{\pm}\},\Re\{\sigma\}]$, is the vector of remaining variables, $M$ is the coefficient matrix, the form of which can be directly inferred from Eqs. (\ref{eq:rhogg-system}) by rewriting them in matrix-vector notation. The constraint in Eq. (\ref{eq:algebraicequation}), on the other hand, establishes the relationships between the $y-$vector variables so that $\Im\{\sigma\} = 0$ is satisfied. The existence and uniqueness of such a solution is a mathematical topic on its own and is out of the scope of this work. However, we have shown that in the special case when $\rho_{gg} = 1/3 = \text{const}$ a solution is possible and is 
in fact stable/unstable:

TODO Investigate the stability of the solution!!!!




Next, let us consider when will the reality condition for the Rabi frequency $\beta(x,t)$  hold. To investigate that, let us take an input pulse $\beta_{in}(t_0) = \beta(x=0,t_0)$, with Fourier transform $F_{in}(\omega)$, entering the medium at the left facet, such that at $F_{in}(\omega)^* = F_{in}(-\omega)$, i.e. $\beta_{in}$ is real. After propagating a distance $L$ inside the cavity the pulse transforms according to \cite{weiner2011ultrafast}
\begin{align}
\label{eq:fout-expansion}
\beta(x=L,t) &= \int_{-\infty}^{\infty} F(x=L,\omega)e^{-i\omega t}d\omega = \int_{-\infty}^{\infty} F_{in}(\omega)e^{-i\omega t}e^{g(\omega)L+i\Psi(\omega)}d\omega,
\end{align}
where $g(\omega)$ denotes the spectral gain per unit length and $\Psi(\omega)=k(\omega)L$, the acquired phase. In order for $\beta(x=L,t)$ to remain real, it is sufficient that $F(x=L,\omega)^*= F(x=L,-\omega)$. To find when this holds, we first expand the wave number $k(\omega)$ around $\omega=0$ up to fourth order to get  
\begin{align}
\label{eq:k-expansion}
k(\omega) = \beta_1\omega + \frac{\beta_2}{2}\omega^2 + \frac{\beta_3}{6}\omega^3 + O(\omega^4), 
\end{align}
which is justified since, by virtue of the rotating wave approximation, we had centred the spectrum around zero and also subtracted out $\beta_0 = k_0$ as the wave number at the central frequency $\omega_0$. Now from Eqs. (\ref{eq:fout-expansion}) and (\ref{eq:k-expansion}), we can write (neglecting terms $\propto O(\omega^4)$)
\begin{subequations}
	\begin{align}
	F(x=L,\omega)^* &\approx F_{in}(\omega)^* \exp\{g(\omega)L-i(\beta_1\omega + \frac{\beta_2}{2}\omega^2 + \frac{\beta_3}{6}\omega^3 )L\}, \\
	F(x=L,-\omega) &\approx F_{in}(-\omega) \exp\{g(-\omega)L+i(-\beta_1\omega + \frac{\beta_2}{2}\omega^2 - \frac{\beta_3}{6}\omega^3 )L\},
	\end{align}
\end{subequations}
which will be equal if
\begin{subequations}
	\begin{align}
	g(-\omega) &= g(\omega), \label{eq:symmetric-gain}\\
	\beta_2 = \beta_4 &= ... = \beta_{2n}=0. \label{eq:vanish-even-order-dispersion}
	\end{align}
\end{subequations}
The symmetric gain condition, Eq. (\ref{eq:symmetric-gain}), will be satisfied when the device is biased at injector $\leftrightarrow$ upper laser level resonance. On the other hand Eq. (\ref{eq:vanish-even-order-dispersion}) requires vanishing even order dispersion, which is approximately satisfied in the tunneling resonance case as then the higher order phase $\Psi$ has equal in magnitude, but opposite in sign, components immediately below and above the central frequency $\omega_0$. For illustrative purposes, we plot the higher order phase, the amplitude gain and the field envelope, calculated with our model in Ref. \onlinecite{petz2016}, in the case when $\epsilon = 0$ (Fig. \ref{fig:dispersion-on-off-resonance}a) and when $\epsilon < 0$ (Fig. \ref{fig:dispersion-on-off-resonance}b). 
\begin{figure}[h!]
	\begin{center}
		\includegraphicsXL{IMGS/dispersion-on-off-resonance.eps}
		\caption{ \textbf{a} Higher order phase $\Psi$ and a third order polynomial fit $\Psi_{fit}$ to it, spectral gain  and the real and imaginary parts of $f$, i.e. $\Re\{f(t)\}$ and $\Im\{f(t)\}$, respectively when $\epsilon = 0$. \textbf{b} Same as \textbf{a}, but when $\epsilon <0$.  } \label{fig:dispersion-on-off-resonance}
	\end{center}	
\end{figure}

\vspacec
Considering the phase $\Psi$ for a second, we see that within the region of interest, i.e. from 3.5 THz to 4.2 THz, in both cases it can be described very well by a third order polynomial fit $\Psi_{fit}(\omega) = \beta_2\omega^2/2 + \beta_3\omega^3/6$ with vanishing $\beta_2$ coefficient in the symmetric case ($\epsilon = 0$) and non-negligible GVD coefficient in the asymmetric case $\epsilon <0$. Simulating the propagation of an initially real-valued Gaussian pulse $f_{in}(t)$ for several round trips, we see that when the symmetric gain and vanishing even order dispersion conditions are satisfied, the field envelope does evolve mainly within the domain of real numbers, as can be verified from the negligibly small imaginary part of $f(t)$ in Fig.~\ref{fig:dispersion-on-off-resonance}a. On the contrary, when the gain symmetry condition is violated, the initially real envelope evolves as a complex number, which is illustrated in Fig.~\ref{fig:dispersion-on-off-resonance}b.

As for the experimental device in Ref. \onlinecite{burghoff2014terahertz}, we note that this laser probably approximately satisfies the $\beta_2 = 0$ condition, due to the incorporation of a dispersion compensation mechanism designed specifically to cancel this coefficient. 

We also would like to mention that the temporal hole burning effect will be present in the assymetric regime, as verified by experiment \cite{burghoff2015evaluating}, however only the perfectly symmetric configuration is subject to the simple theoretical treatment presented in this work. For a more realistic interpretation, it is essential to also consider the effects of detuning onto the pulse switching behaviour, which unfortunately is out of the scope of this publication.

Lastly, the second assumption we made was that the dipole moments have equal algebraic value, i.e. $d_{+} = d_{-} = d$. What if they have opposite signs, i.e. $d_{+} = -d_{-} = d$? It is very easy to show that one can derive an equivalent quasi-two level system, but this time with the coherence set as $\eta = \eta_{+g}-\eta_{-g}^*$, instead of the substitution made above. In this case, the interpretation of the pulse switching behaviour remains the same, as the reader can readily verify by him/her-self.


\bibliographystyle{ieeetr}

	
	
	
	
\bibliography{D:/docs/MAIN-PROJECTS/PAPERS/literature/bib_resources.bib}
\end{document}

