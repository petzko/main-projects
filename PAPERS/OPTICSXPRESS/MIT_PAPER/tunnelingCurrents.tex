%%%%%%%%%%%%%%%%%%%%%%%%%%%%%%%%%%%%%%%%%%%%%%%%%%%%%%%
%                File: OpEx_temp.tex                  %
%             Created: 2 September 2009               %
%                Updated: 15 May 2015                 %
%                                                     %
%           LaTeX template file for use with          %
%           OSA's journals Optics Express,            %
%             Biomedical Optics Express,              %
%            and Optical Materials Express            %
%                                                     %
%  send comments to Theresa Miller, tmiller@osa.org   %
%                                                     %
% This file requires style file, opex3.sty, under     %
%              the LaTeX article class                %
%                                                     %
%   \documentclass[10pt,letterpaper]{article}         %
%   \usepackage{opex3}                                %
%                                                     %
%                                                     %
%       (c) 2015 Optical Society of America           %
%%%%%%%%%%%%%%%%%%%%%%%%%%%%%%%%%%%%%%%%%%%%%%%%%%%%%%%

%%%%%%%%%%%%%%%%%%%%%%% preamble %%%%%%%%%%%%%%%%%%%%%%%%%%%
\documentclass[10pt,letterpaper]{article}
\usepackage{opex3}
\usepackage{color}
\usepackage[latin9]{inputenc}
\usepackage{mathrsfs,amsmath}
\usepackage{graphicx}%
\usepackage{float}
\usepackage{amsfonts}%
\usepackage{amssymb}
\usepackage{braket}
\usepackage{bm}
\newcommand{\mb}[1]{\bm{#1}}
\usepackage[T1]{fontenc}



\def\Nabla{\bm{\nabla}}
\def\bm{\mathbf}
\def\curl{\Nabla\times}
\def\div{\Nabla\cdot}
\def\lap{\Delta}
\def\vlap{\Delta}
\def\x{\hat{e}_{x}}
\def\y{\hat{e}_{y}}
\def\z{\hat{e}_{z}}
\DeclareMathOperator{\Tr}{Tr}
\def\p{\partial}
\bibliographystyle{ieeetr}
%%%%%%%%%%%%%%%%%%%%%%% begin %%%%%%%%%%%%%%%%%%%%%%%%%%%%%%
\begin{document}

%%%%%%%%%%%%%%%%%% title page information %%%%%%%%%%%%%%%%%%
\title{On the tunneling currents from a pair of conduction channels}

\author{PeTz$^{1*}$}
\address{$^1$ Institute for Nanoelectronics, Technische Universit�t M�nchen, D-80333 Munich, Germany}

\email{*petar.tzenov@tum.de} %% email address is required

% \homepage{http:...} %% author's URL, if desired

%%%%%%%%%%%%%%%%%%% abstract and OCIS codes %%%%%%%%%%%%%%%%
%% [use \begin{abstract*}...\end{abstract*} if exempt from copyright]

\begin{abstract}
	We will consider the current density through a pair of quantum wells coupled via two resonant tunneling channels with different detuning energy. We will derive analytical expressions for the tunneling currents via the classical density matrix formalism and compare the relative magnitudes of electron flow along both channels. We will also investigate the onset of quantum interferences and their effects on the measured current density.     
\end{abstract}
\ocis{(000.0000) General.} % REPLACE WITH CORRECT OCIS CODES FOR YOUR ARTICLE, MINIMUM OF TWO; Avoid using the OCIS codes for “General” or “General science” whenever possible.
%%%%%%%%%%%%%%%%%%%%%%% References %%%%%%%%%%%%%%%%%%%%%%%%%
\bibliography{literature}

%%%%%%%%%%%%%%%%%%%%%%%%%%  body  %%%%%%%%%%%%%%%%%%%%%%%%%%
\section{Outline}

As a toy model we will take a quantum well heterostructure comprised of two wells, which we will denote by $L$ and $R$ indices for the left and right well, respectively, and three quantized electron states, $\Psi_{L1}$ , $\Psi_{L2}$ and $\Psi_R$ for a pair of levels in the left well and a single level in the right, respectively, which we will also assume to be completely localized in each respective well, if calculated  within the tight binding basis. 

The task of our derivation is to investigate the dependence of the relative strengths of the tunneling currents through the $L1 \leftrightarrow R$ and $L2 \leftrightarrow R$ tunneling channels, which form after the wells have been brought to interact. Lastly, as an example of a system where this discussion might be of interest, we can point out resonant LO phonon quantum cascade lasers, where the electron injection into the active region is usually realized namely via resonant tunneling, and structures with a pair of injector levels have shown to outperform similar counterparts with single level. 

We will begin the treatment within the tight-binding basis where the wave functions of each well are calculated separately, and the coupling, which occurs when the wells are brought close to each other, is realized via introducing a "coupling" energy inside the Hamiltonian. In Dirac notation the Hamiltonian operator (in the tight binding basis) can be written as
\begin{align}
\hat{H} & = E_{L1}\Ket{1}\Bra{1} + E_{2}\Ket{2}\Bra{2}+E_{R}\Ket{R}\Bra{R}  \nonumber \\
		&+ W_{1} \big(\Ket{1}\Bra{R} +\Ket{R}\Bra{1} \big) + W_{2} \big(\Ket{2}\Bra{R} +\Ket{R}\Bra{2} \big). 
\end{align}

In matrix notation this yields

\begin{align}
\begin{pmatrix} 
	E_1 & 0 & W_1 \\
	0  & E_2 &  W_2 \\
	W_1  & W_2 & E_R   
	\end{pmatrix} 
\end{align}

The time evolution of an electron with this Hamiltonian is governed by the von Neumann equation
\begin{equation}
\frac{d \hat{\rho}}{dt} = \frac{i}{\hbar}[\hat{\rho},\hat{H}],
\end{equation}
where $\hat{\rho}$ is the usual density matrix operator. On the other hand the current density through the so isolated system is proportional to the expectation value of the velocity operator:
\begin{equation}
J = e N \Braket{\hat{v}},
\end{equation}
with $\hat{v} = \frac{\hat{p}}{m} = \frac{i}{\hbar} [\hat{H},\hat{x}]$, where we have used that, due to the canonical commutation relations $[\hat{x},\hat{p}_x] = i\hbar$, it follows $[\hat{H},\hat{x}] = -\frac{i\hbar}{m}\hat{p}$. Above, $e$ denotes the carrier charge, $N$ the average volume concentration and $\hat{x}$ the position operator.
\section{Real eigenenergies}
In the tight binding basis, we can safely assume that the only non-zero element of the position matrix $X_{ij} = \Bra{i}\hat{x}\Ket{j}$ is the one for $i = 1,2$ and $j=2,1$. Thus the commutator $\hat{H},\hat{x}$ becomes, in matrix form
\begin{align}
[H,X] &= \left[ \begin{pmatrix} 
E_1 & 0 & W_1 \\
0  & E_2 &  W_2 \\
W_1  & W_2 & E_R =0
\end{pmatrix},
\begin{pmatrix} 
0 & \mu/e & 0 \\
\mu/e  & 0 &  0 \\
0  & 0 & 0   
\end{pmatrix}
\right] \nonumber \\
&= \frac{1}{e}
\begin{pmatrix} 
E_1 & 0 & W_1 \\
0  & E_2 &  W_2 \\
W_1  & W_2 & 0
\end{pmatrix} \begin{pmatrix} 
0 & \mu & 0 \\
\mu  & 0 &  0 \\
0  & 0 & 0   
\end{pmatrix} - \frac{1}{e}\begin{pmatrix} 
0 & \mu & 0 \\
\mu  & 0 &  0 \\
0  & 0 & 0   
\end{pmatrix}
\begin{pmatrix} 
E_1 & 0 & W_1 \\
0  & E_2 &  W_2 \\
W_1  & W_2 & 0
\end{pmatrix} \nonumber \\
&= \frac{1}{e}\begin{pmatrix} 
0 & -\mu(E_2-E_1) & -\mu W_2 \\
\mu(E_2-E_1)  & 0 &  -\mu W_1 \\
\mu W_2  & \mu W_1 & 0   
\end{pmatrix},
\end{align}
where $\mu$ is the dipole element between the states in the left well. With the density matrix we can obtain the expectation value of $\hat v$ and thus the current density as
\begin{align}
J &= e N \braket{\hat v} = eN \frac{i}{\hbar} \Tr\left \{\rho \cdot [ H, X]\right \} \nonumber \\
  &= i \frac{\mu N}{\hbar} \Tr\left \{
  \begin{pmatrix}
  \rho_{11}& \rho_{12} & \rho_{1R} \\
  \rho_{21} & \rho_{22} & \rho_{2R} \\ 
  \rho_{R1} & \rho_{R2} & \rho_{RR}
  \end{pmatrix} \cdot  \begin{pmatrix} 
  0 & -(E_2-E_1) & - W_2 \\
  (E_2-E_1)  & 0 &  - W_1 \\
   W_2  &  W_1 & 0   
  \end{pmatrix}\right \}.
\end{align}
 This gives
 \begin{align}
 J =  i \frac{\mu N}{\hbar} \left[ \left(E_2-E_1 \right) (\rho_{12} - \rho_{21}) + W_2(\rho_{1R}-\rho_{R1}) +  W_1(\rho_{2R}-\rho_{R2})\right].
  \end{align}

To obtain the formulas for the current density we need to solve the DM EOM in steady state. The von Neumann equation reads:
\begin{align}
\frac{d\rho_{lm}}{dt} = \frac{i}{\hbar} [\rho,H]_{lm} = \frac{i}{\hbar}\sum_{k \in \{1,2,R\}}\rho_{lk}H_{km} - \rho_{km}H_{lk}
\end{align}
thus the coherence elements become:
$$
\frac{d\rho_{12}}{dt} = \frac{i}{\hbar} ( -\rho_{12}(E_1-E_2)+\rho_{1R}W_2-\rho_{R2}W_1),
$$
$$
\frac{d\rho_{1R}}{dt} = \frac{i}{\hbar} (  -\rho_{1R}E_1 +(\rho_{11}-\rho_{RR})W_1+ \rho_{12}W_2),
$$
$$
\frac{d\rho_{2R}}{dt} = \frac{i}{\hbar} (-\rho_{2R}E_2+(\rho_{22}-\rho_{RR})W_2+\rho_{21}W_1).
$$

Let us recast the above equation into matrix form:
\begin{align}
\bm{\dot y} = \frac{i}{\hbar} \left [ A_1 \bm{y} + A_2 \bm{y}^\dagger +\bm{b}\right], 
\end{align}
where $A_1,A_2$ are 3$\times$3 real matrices and $\bm{b}$ is a real vector. We have also set $\bm{y} = [\rho_{12},\rho_{1R},\rho_{2R}]^T $ as the coherence terms vector and $\dagger$ denotes the complex conjugate. 
\begin{align}
A_1 = \begin{pmatrix}
-(E_1-E_2)& W_2 & 0 \\
W_2 & -E_1 & 0 \\ 
0 & 0 & -E_2
\end{pmatrix}
\end{align}
\begin{align}
A_2 = \begin{pmatrix}
0& 0 & -W_1 \\
0 & 0 & 0 \\ 
W_1 & 0 & 0
\end{pmatrix}
\end{align}
\begin{align}
\textbf{b} = \begin{pmatrix}
0\\
W_1 (\rho_{11}-\rho_{RR})\\
W_2 (\rho_{22}-\rho_{RR})
\end{pmatrix}
\end{align}

Let us divide the vector $\bm{y}$ into its real and imaginary parts, $ \bm{y}= \bm{y}_{re}+i\bm{y}_{im}$. Then plugging into the above equation we obtain:
\begin{align}
\bm{\dot y}_{re} + i\bm{\dot y}_{im} &= \frac{i}{\hbar} \left [ A_1 (\bm{y}_{re}+i\bm{y}_{im})  + A_2 (\bm{y}_{re}-i\bm{y}_{im})+\bm{b}\right], \nonumber \\
&= -\frac{1}{\hbar}(A_1-A_2)\bm{y}_{im} +\frac{i}{\hbar} [(A_1+A_2)\bm{y}_{re} + \bm{b}].
\end{align}
Thus we obtain two coupled ODEs:
\begin{align}
\bm{\dot y}_{re} &= -\frac{1}{\hbar}(A_1-A_2)\bm{y}_{im} , \\
\bm{\dot y}_{im} &= +\frac{1}{\hbar} (A_1+A_2)\bm{y}_{re}+ \frac{1}{\hbar} \bm{b} (=\tilde{\bm b}). 
\end{align}
Eliminating $y_{re}$ from the second equation we get:
\begin{align}
\bm{\ddot y}_{im} &= -\frac{1}{\hbar^2} (A_1+A_2)(A_1-A_2)\bm{y}_{im} +\frac{1}{\hbar} \bm{\dot b}= (\frac{iW}{\hbar})^2\bm{y}_{im}  +\frac{1}{\hbar} \bm{\dot b},
\end{align}
where $W = \sqrt{(A_1+A_2)(A_1-A_2)} = \sqrt{A_1^2-A_2^2 - [A_1,A_2]}$. 

Back to our ODE,  it is subject to the initial conditions 
\begin{align}
	\bm{y}_{im}(t=0) &= \bm{y}_{im}^0 \nonumber \\
	\bm{\dot{y}}_{im}(t=0) &= \frac{1}{\hbar} (A_1+A_2)\bm{y}_{re}^0+ \frac{1}{\hbar}\bm{b}^0,\nonumber \\
\end{align}
where $\bm{y}^0=\bm{y}_{re}^0+i\bm{y}_{im}^0$ is the initial coherence vector of the system. We also have
\begin{align}
	A_1^2 = \begin{pmatrix}
		(E_1-E_2)^2+W_2^2 & -(2E_1-E_2)W_2  & 0 \\
		 -(2E_1-E_2)W_2 & W_2^2+E_1^2 & 0 \\ 
		0 & 0 & E_2^2
	\end{pmatrix},
\end{align}
\begin{align}
	A_2^2 = \begin{pmatrix}
		-W_1^2& 0  & 0 \\
		0 & 0 & 0 \\ 
		0 & 0 & -W_1^2
	\end{pmatrix},
\end{align}
\begin{align}
	[A_1,A_2] = \begin{pmatrix}
	0& 0  & W_1(E_1-2E_2) \\
	0 & 0 & W_1W_2 \\ 
	W_1(E_1-2E_2) & W_1W_2 & 0
	\end{pmatrix},
\end{align}
which yields
\begin{align}
	W^2 = A_1^2-A_2^2-[A_1,A_2] = \begin{pmatrix}
		(E_1-E_2)^2+W_2^2+W_1^2& -(2E_1-E_2)W_2  & -W_1(E_1-2E_2) \\
		-(2E_1-E_2)W_2 & W_2^2+E_1^2 & -W_1W_2 \\ 
		-W_1(E_1-2E_2) &-W_1W_2 & E_2^2+W_1^2
	\end{pmatrix}.
\end{align}
Now, we immediately see that since $W^2$ is symmetric,i.e. Hermitian, then so is $W$ and furthermore their eigenvalues are related by $eig(W)_i = \sqrt{eig(W^2)_i}$ due to the fact that if $W^2 = VDV^{-1}$ with $V$ unitary and $D$ diagonal then it follows that $W = VD^{1/2}V^{-1}$. The latter representation is possible, namely due to the fact that $W^2$ is Hermitian. 

The above equation has analytical solution given by the formula:
\begin{align}
\bm{y}_{im}(t) &= e^{\frac{i}{\hbar}Wt}\bm{c}_1+e^{-\frac{i}{\hbar}Wt}\bm{c}_2 + \bm{y}_p(t),
\end{align}
with $\bm{c}_1$ and $\bm{c}_2$ to be determined from the initial conditions and $y_p$ a particular solution of the second order ODE (in the case when $\bm{b}(t)$ is time independent then $y_p(t) \equiv 0$).  The constant vectors are the solution of the system:
\begin{align}
\bm{c}_1 + \bm{c}_2+\bm{y}_p(0) &= \bm{y}_{im}^0, \\
\frac{iW}{\hbar}[\bm{c}_1 - \bm{c}_2] +\bm{\dot{y}}_p(0) &= \frac{1}{\hbar} (A_1+A_2)\bm{y}_{re}^0+ \frac{1}{\hbar}\bm{b}^0.
\end{align}
In case $\bm{b}(t) = const$ , then $\bm{y}_p = 0$ and the constants are given as  
\begin{align}
\bm{c}_1 &= \frac{1}{2}\bm{y}_{im}^0-\frac{i}{2}W^{-1}(A_1+A_2)\bm{y}_{re}^0-\frac{i}{2}W^{-1}\bm{b}, \\
\bm{c}_2 &= \frac{1}{2}\bm{y}_{im}^0+\frac{i}{2}W^{-1}(A_1+A_2)\bm{y}_{re}^0+\frac{i}{2}W^{-1}\bm{b}. \\
\end{align}
Also, assuming initial $\bm{y}^0 = \bm{0}$ we simplify further
\begin{align}
\bm{c}_1 &= -\frac{i}{2}W^{-1}\bm{b}, \\
\bm{c}_2 &= +\frac{i}{2}W^{-1}\bm{b}, \\
\end{align}
which yields the solution
\begin{align}
\bm{y}_{im}(t) &= -\frac{i}{2}[e^{\frac{i}{\hbar}Wt} - e^{\frac{-i}{\hbar}Wt}]W^{-1}\bm{b} =  \sin(\frac{W}{\hbar}t) W^{-1}\bm{b},\\
\end{align}
where the $\sin(\frac{W}{\hbar}t)$ is defined via its Taylor series expansion.

Keeping the above analytical solution in mind, we can and re-cast the expression for $J$ it into our vector notation to get

\begin{align}
J(t) &= -\frac{2\mu N}{\hbar} \begin{pmatrix} 1 & 1 & 1 \end{pmatrix} \times \begin{pmatrix} 
(E_2-E_1) & 0  & 0 \\
0 & W_2 &  0 \\
0  & 0 & W_1   
\end{pmatrix}  _{im} = -\frac{2\mu N}{\hbar} \bm{e}^T \Omega \bm{y}_{im}.
\end{align}
With the final form for $J(t)$ given as:
\begin{align}
J(t) &= -\frac{2\mu N}{\hbar}\bm{e}^T \Omega W^{-1}\sin(\frac{W}{\hbar}t) \bm{b}.
\end{align}
Let $\hbar\omega_1,\bm{v}_1$,$\hbar\omega_2,\bm{v}_2$ and $\hbar\omega_3,\bm{v}_3$ be the eigenvalue-eigenvector pairs of the Matrix $W$. Then in this orthonormal basis, $\bm{b}$ is given by:
\begin{align}
\bm{b} &= \alpha_1\bm{v_1} +\alpha_2\bm{v_2}+\alpha_1\bm{v_3}, \nonumber \\
\alpha_j &= \bm{v}_j\cdot \bm b,
\end{align}
which gives us the following expression for the current density $J$
\begin{align}
J(t) &= -\frac{2\mu N}{\hbar}\bm{e}^T \Omega [\frac{\alpha_1}{\omega_1}\sin(\omega_1 t) \bm{v}_1+\frac{\alpha_2}{\omega_2}\sin(\omega_2 t) \bm{v}_2+\frac{\alpha_3}{\omega_3}\sin(\omega_3 t) \bm{v}_3] \nonumber \\
&=-\frac{2\mu N}{\hbar}\sum_{k} \frac{\alpha_k}{\omega_k}\sin(\omega_k t) \sum_{j} \Omega_{jj}  v_{jk}.
\end{align}
Intuitively, we can see that the current density through the barrier is given as a superposition of three sinusoidal oscillations with well defined frequencies (the eigenvalues of the $W$ matrix) and amplitudes proportional to $\propto \alpha_k/\omega_k$, with $\alpha_k$ essential determined by the pumping/depopulation scheme -> this parameter can be customized as desired. 

\section{Complex eigenenergies}
The case when $A_{1/2}$ have complex coefficients (i.e. dephasing included). Then the original ODE can be rewritten as
\begin{align}
\bm{\dot y} &= \frac{i}{\hbar} \left [ A_1 \bm{y} + A_2 \bm{y}^\dagger +\bm{b}\right] 
		     = \frac{i}{\hbar}[(A_1^{re}+iA_1^{im})(\bm{y}_{re}+i\bm{y}_{im}) +(A_2^{re}+iA_2^{im})(\bm{y}_{re}-i\bm{y}_{im}) + \bm b ] \nonumber \\
			&= \frac{i}{\hbar}[(A_1^{re}+A_2^{re})\bm{y}_{re}-(A_1^{im}-A_2^{im})\bm{y}_{im} + i \big((A_1^{im}+A_2^{im})\bm{y}_{re} + (A_1^{re}-A_2^{re})\bm{y}_{im} \big)+\bm{b}]
\end{align}
Therefore
\begin{align}
\bm{\dot y}_{re} &= -\frac{1}{\hbar} \big((A_1^{im}+A_2^{im})\bm{y}_{re} + (A_1^{re}-A_2^{re})\bm{y}_{im} \big) \label{eq:dotyre}\\
\bm{\dot y}_{im} &= +\frac{1}{\hbar} \big((A_1^{re}+A_2^{re})\bm{y}_{re}-(A_1^{im}-A_2^{im})\bm{y}_{im} +b \big) \label{eq:dotyim} 
\end{align}
Again differentiating once more $\bm{\dot{y}}_im$ we obtain:
\begin{align}
\label{eq:ddotyim}
\bm{\ddot y}_{im} &= +\frac{1}{\hbar} \big((A_1^{re}+A_2^{re})\bm{\dot{y}}_{re}-(A_1^{im}-A_2^{im})\bm{\dot{y}}_{im} \big) \nonumber \\
				  &= +\frac{1}{\hbar} \big((A_1^{re}+A_2^{re})\left[    -\frac{1}{\hbar} \big((A_1^{im}+A_2^{im})\bm{y}_{re} + (A_1^{re}-A_2^{re})\bm{y}_{im} \big)  \right]  -(A_1^{im}-A_2^{im})\bm{\dot{y}}_{im} \big) \nonumber \\
				  &= -\frac{1}{\hbar^2} (A_1^{re}+A_2^{re})(A_1^{im}+A_2^{im})\bm{y}_{re} -\frac{1}{\hbar^2} (A_1^{re}+A_2^{re})(A_1^{re}-A_2^{re})\bm{y}_{im}-\frac{1}{\hbar}(A_1^{im}-A_2^{im})\bm{\dot{y}}_{im}.
\end{align}

Note that in our approach the matrix $A_1^{im}$ is diagonal whereas the matrix $A_2^{im}$ is actually a zeroes matrix, since the coupling energies $W_1$ and $W_2$ are purely real. Therefore the product $(A_1^{re}+A_2^{re})(A_1^{im}+A_2^{im}) = (A_1^{im}+A_2^{im})(A_1^{re}+A_2^{re})$, i.e. it actually commutes, which allows us to substitute in the above equation $(A_1^{re}+A_2^{re})\bm{y}_{re}/\hbar$ with the expression
\begin{align}
\frac{1}{\hbar}(A_1^{re}+A_2^{re})\bm{y}_{re} = \bm{\dot y}_{im}+\frac{1}{\hbar}(A_1^{im}-A_2^{im})\bm{y}_{im}-\frac{\bm{b}}{\hbar},
\end{align}
to completely eliminate $\bm{y}_{re}$ from Eq. (\ref{eq:ddotyim}). 

\begin{align}
\label{eq:ddotyim}
\bm{\ddot y}_{im} &= -\frac{1}{\hbar} (A_1^{im}+A_2^{im})\left[  \bm{\dot y}_{im}+\frac{1}{\hbar}(A_1^{im}-A_2^{im})\bm{y}_{im}-\frac{\bm{b}}{\hbar} \right] 
					 -\frac{1}{\hbar^2} (A_1^{re}+A_2^{re})(A_1^{re}-A_2^{re})\bm{y}_{im}-\frac{1}{\hbar}(A_1^{im}-A_2^{im})\bm{\dot{y}}_{im} \nonumber \\
				  &= \frac{2}{\hbar} A_1^{im}\bm{\dot y}_{im} +((\frac{iW^{re}}{\hbar})^2 + (\frac{iW^{im}}{\hbar})^2)\bm{y}_{im}+\frac{1}{\hbar^2}(A_1^{im}+A_2^{im})\bm{b},
\end{align}
where $W^{re} = \sqrt{(A_1^{re}+A_2^{re})(A_1^{re}-A_2^{re})} $ and $W^{im} = \sqrt{(A_1^{im}+A_2^{im})(A_1^{im}-A_2^{im})} $. 

Setting $\frac{1}{\hbar}A_1^{im} = \Gamma$, $\frac{1}{\hbar^2}((W^{re})^2)+(W^{im})^2  = \Delta$ and $\frac{1}{\hbar^2}(A_1^{im}+A_2^{im})\bm{b} = \bm{f}$ we obtain the equation
\begin{align}
\label{eq:ddotyimHO}
\bm{\ddot y}_{im} &= 2\Gamma\bm{\dot y}_{im} -\Delta\bm{y}_{im}+\bm{f},
\end{align}
which is nothing but the equation of a dampened harmonic oscillator driven by a constant force $\bm{f}$. To find the solution take the ansatz
$$
\bm{y}_{im} = e^{Qt}\bm{\chi} + \bm{c}, 
$$
where $\bm{\chi}$ and $\bm{c}$ are constant vectors to be determined and $Q$ is a matrix. The characteristic equation for Eq. (\ref{eq:ddotyimHO}) can be now written as
\begin{align}
\label{eq:ddotyimHO}
[Q^2-2\Gamma Q+\Delta]\chi = \bm{0},
\end{align}
with the solutions
$$
Q_{\pm} = \Gamma \pm\sqrt{\Gamma^2-\Delta} = \Gamma\pm i\sqrt{\Delta-\Gamma^2}. 
$$
Expanding the term $\Delta-\Gamma^2$, using the fact that $A_2^{im} = 0$ it follows that $\Delta-\Gamma^2 = \frac{1}{\hbar^2}(W^{re})^2 = \frac{1}{\hbar^2}W^2$. 

Therefore the general solution is given via:
\begin{align}
\label{eq:yimGENERAL}
\bm{y}_{im} =  e^{(\Gamma+\frac{i}{\hbar}W)t}\bm{\chi^+}+ e^{(\Gamma-\frac{i}{\hbar}W)t}\bm{\chi^-} + {\Delta^{-1}}\bm{f}, 
\end{align}
and again the boundary conditions
\begin{align}
\bm{y}_{im}(t=0) &= y_{im}^{0} \\
\bm{\dot y}_{im}(t=0) &= +\frac{1}{\hbar} \big((A_1^{re}+A_2^{re})\bm{y}_{re}^{0}-(A_1^{im}-A_2^{im})\bm{y}_{im}^{0} +b \big) \\
\end{align}











\end{document}
