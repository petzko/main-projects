%% This document created by Scientific Word (R) Version 3.5
\documentclass[10pt,english,fleqn]{article}%
\usepackage[latin9]{inputenc}
\usepackage{amsmath}
\usepackage{graphicx}%
\usepackage{float}
\usepackage{amsfonts}%
\usepackage{amssymb}
\usepackage{braket}
\usepackage{bm}
\usepackage{color}
%\usepackage{algorithm2e}
%\newcommand{\mb}[1]{\mathbf{#1}} % undergraduate algebra version
%\newcommand{\mb}[1]{#1} % pure math version
\newcommand{\mb}[1]{\bm{#1}}
\usepackage[T1]{fontenc}
\begin{document}
\section{Outline}
\label{sec:outline}
In the following I will derive the Maxwell-Bloch equations for a 3 level quantum cascade laser with one optical transition and resonant tunnelint transport 
effects included. In general, we will assume a resonant phonon design QCL with 3 relevant levels per period, $\Ket{3} , \Ket{2}, \Ket{1}$, where the levels are (in the same order) 
the upper laser level, the lower laser level and the depopulation/injector level of the next period (see \cite{kumar2009}). We will also assume a non-negigible 
anticrossing energy between the injector level ($\Ket{1'}$) of the previous period and the upper laser level ( level $\Ket{3}$ ) of the current period
, $\hbar \Omega_{1'3}$. Taking a biased based approach (i.e. not including explicitly the input current density) we will reduce the 4 level system: 
$\{\Ket{1'}, \hbar \omega_{1'} \} ,\{\Ket{3}, \hbar \omega_{3} \} ,\{\Ket{2}, \hbar \omega_{2} \} ,\{\Ket{1}, \hbar \omega_{1} \}$, to the 3 level system 
$\{\Ket{1'}, \hbar \omega_{1'} \} ,\{\Ket{3}, \hbar \omega_{3} \} ,\{\Ket{2}, \hbar \omega_{2} \}$, by employing periodic bdry conditions to 
ensure current continuity (see Sec. \ref{sec:relaxation-terms}). 

Following the standard derivation procedures of the MB equations we will employ a density matrix approach to describe the evolution of our statistical ensemble 
of electrons, coupled via a polarization term to a classical wave propagation equation. 

Within the densitity matrix formalism, we can write the equations of motion as follows:
\begin{align}
 \label{eq:luiville-eqn}
 i\hbar \frac{d \rho}{dt} = [H;\rho] ,
\end{align}
where $\rho$ is the density matrix: 
\begin{align}
 \label{eq:dmatrix-define}
\rho = \begin{bmatrix}
\rho_{1'1'}& \rho_{1'3} & \rho_{1'2} \\
\rho_{31'} & \rho_{33} & \rho_{32} \\
\rho_{21'} & \rho_{23} & \rho_{22}
\end{bmatrix} , 
\end{align}
$H = H_0 + H_{RT} + H_I$ is the Hamiltonian of the system, consisting of 3 different components corresponding to the unperturbed Hamiltonian,
the resonant tunneling interaction hamiltonian and the perturbation energy due to the electric field, respectively. 
\begin{align}
 \label{eq:hamiltonians}
H_0 = 
\begin{bmatrix}
\hbar\omega_{1'} & 0 & 0 \\
0  & \hbar\omega_{3} &  0 \\
0  & 0  & \hbar\omega_{2}   
\end{bmatrix} , 
H_{RT} = \begin{bmatrix}
0  & \hbar\Omega_{1'3}  &  0 \\
\hbar\Omega_{1'3}  & 0  & 0 \\
0  & 0  & 0    
\end{bmatrix} , 
H_{I} =  \begin{bmatrix}
0  & 0  &  0 \\
0 &  0  & E\mu_{32} \\
0  & E\mu_{23} &  0     
\end{bmatrix}, 
\end{align}
where $\mu_{ij} = |e|\Bra{i} \mb{r} \ket{j}$ is the dipole interaction between electrons in the $i^{\text{th}}$ and  $j^{\text{th}}$ subbands. Notice that we 
have multiplied by the magnitude of the elementary electron charge $|e|$, since the - sign of the electron charge canceles with the negative sign, conventionally
put in front of the potential energy term of an electric dipole. 

Other pertrubations due to spontaneous emission, LO-phonon scattering, e-e scattering etc., the so called collision terms, will be included phenomenologically with the aid of relaxation terms 
(see Sec. \ref{sec:relaxation-terms} ). 

In this context, the density matrix equation (Eq. \ref{eq:luiville-eqn}) reads: 
\begin{align}
\frac{d \rho_{1'1'}}{d t} &= i\Omega_{1'3} (\rho_{1'3} - \rho_{31'}) + coll_{1'1'} \\ 
\frac{d \rho_{33}}{d t}   &= i\Omega_{1'3} (\rho_{31'} - \rho_{1'3}) + i\frac{\mu_{32} E}{\hbar} (\rho_{32}-\rho_{23}) + coll_{33} \\
\frac{d \rho_{22}}{d t}   &= i\frac{\mu_{32} E}{\hbar} (\rho_{23}-\rho_{32}) + coll_{22} \\
\frac{d \rho_{31'}}{d t}  &= i\omega_{1'3}\rho_{31'} -i \Omega_{1'3}(\rho_{1'1'} - \rho_{33}) -i\frac{\mu_{32}E}{\hbar}\rho_{21'} +coll_{31'}  \\
\frac{d \rho_{23}}{d t}   &= i\omega_{32}\rho_{23} -i \frac{\mu_{32}E}{\hbar}(\rho_{33}-\rho_{22}) +i\Omega_{1'3}\rho_{21'} +coll_{23'}  \\
\frac{d \rho_{21'}}{d t}  &= i\omega_{1'2}\rho_{21'} -i\frac{\mu_{32}E}{\hbar}\rho_{31'} +i\Omega_{1'3}\rho_{23} +coll_{21'} 
\end{align}

Lastly, we write down the polarization as the expectation of the electric dipole moment, i.e. :
\begin{equation} P(x,t) = -N\Gamma Tr\{\mb\rho \mb\mu\} = -N\Gamma(\mu_{32}\rho_{32} + \mu_{23}\rho_{23}).\end{equation} Here,
one must not forget the minus sign carried by the negative electron charge and the fact that we have taken $\mb{\mu} = |e|\mb{r}$.
Then, the wave equation is written as follows:
\begin{equation}
(\partial^2_{x} -\frac{n^2}{c^2}\partial^2_t) E = \frac{1}{\epsilon_0 c^2}\partial^2_t P,
\end{equation}

\section{Relaxation Terms}
\label{sec:relaxation-terms}


Similarly to \cite{bidegaray2001} one has to make sure to observe certain
constraints when including the physical relaxation rates, or also known as collision terms. As it was already mentioned in Sec. \ref{sec:outline}, 
we model our resonant-phonon design as a four level system with 3 relevant levels from the same period coupled to an additional "injector level" ($\Ket{1'}$) 
from the previous period. When the structure is biased close to $1'-3$ resonance, electrons in state $\Ket{1'}$ tunnel through the injection barrier into the upper
laser level $\Ket{3}$. From there they scatter radiateively to the lower laser level ($\Ket{2}$) and hopefully quickly deexcite to the depopulation level 1, via 
optical phonon relaxation mechanisms. Due to the periodicity of the structure, this depopulation level is in fact the injector level of the next period and again 
the same procedure is repeated until the current is extracted out of the heterostructure. As a typical QCL active region consists of many (80-100) 
periods it becomes impractical to model these in a self consistent manner. An alternative way is to take one 3 level system and impose on it periodic 
boundary conditions, and namely $\rho_{1'1'}(t) = \rho_{11}(t)$, for each time $t$. This renders one of the levels (the depopulation level) obsolete and allows
us to greatly reduce the complexity of our equations. This strategy is implemented through the collision terms of the diagonal elements (i.e. populations) of the
density matrix, and namely:
\begin{align}
 coll_{1'1'} &= (\frac{1}{\tau_{31'}}  + \frac{1}{31})\rho_{33} + (\frac{1}{\tau_{21'}}  + \frac{1}{21})\rho_{22}
 - (\frac{1}{\tau_{1'3}} + \frac{1}{\tau_{13}} + \frac{1}{\tau_{1'2}} + \frac{1}{\tau_{12}})\rho_{1'1'}, \\
 coll_{33} &= (\frac{1}{\tau_{1'3}}  + \frac{1}{13})\rho_{1'1'} + \frac{1}{\tau_{23}}\rho_{22}
 - (\frac{1}{\tau_{31'}} + \frac{1}{\tau_{32}} + \frac{1}{\tau_{31}} )\rho_{33}, \\
 coll_{22} &= (\frac{1}{\tau_{1'2}}  + \frac{1}{12})\rho_{1'1'} + \frac{1}{\tau_{32}}\rho_{33}
 - (\frac{1}{\tau_{21'}} + \frac{1}{\tau_{23}} + \frac{1}{\tau_{21}} )\rho_{33}, 
\end{align}
where we have included the relaxation rates to and from level $\Ket{1}$, to ensure that the trace of $\rho$ will be perserved throughout the simulation.
To incorporate the relaxation terms for the off-diagonal elements $\rho_{ij}$ with $i \neq j$, we can write: 
\begin{align}
coll_{ij} = \gamma_{ij} \rho_{ij} , \text{ where } i \neq j \text{ and } \gamma_{ij} = \frac{1}{2} \big( \frac{1}{\tau_i} + 
\frac{1}{\tau_j} \big) + \frac{1}{\tau_{deph}}
\end{align}

Finally our system of equations becomes:  
\begin{align}
\label{eq:mblochbase-start}
\frac{d \rho_{1'1'}}{d t} &= i\Omega_{1'3} (\rho_{1'3} - \rho_{31'}) + (\frac{1}{\tau_{31'}} + \frac{1}{\tau_{31}})\rho_{33}  \\ 
& + (\frac{1}{\tau_{21'}} + \frac{1}{\tau_{21}})\rho_{22} - (\frac{1}{\tau_{1'3}} + \frac{1}{\tau_{13}} + \frac{1}{\tau_{1'2}} + \frac{1}{\tau_{12}})\rho_{1'1'} \\
\frac{d \rho_{33}}{d t}   &= i\Omega_{1'3} (\rho_{31'} - \rho_{1'3}) + i\frac{\mu_32 E}{\hbar} (\rho_{32}-\rho_{23})+ (\frac{1}{\tau_{1'3}} + \frac{1}{\tau_{13}})\rho_{1'1'} \nonumber \\ 
& +  \frac{1}{\tau_{23}}\rho_{22} - (\frac{1}{\tau_{31'}} + \frac{1}{\tau_{32}} + \frac{1}{\tau_{31}}) \rho_{33} \\
\frac{d \rho_{22}}{d t}   &= i\frac{\mu_32 E}{\hbar} (\rho_{23}-\rho_{32}) + (\frac{1}{\tau_{1'2}} + \frac{1}{\tau_{12}})\rho_{1'1'} \\ 
& +  \frac{1}{\tau_{32}}\rho_{33} - (\frac{1}{\tau_{21'}} + \frac{1}{\tau_{23}} + \frac{1}{\tau_{21}}) \rho_{22} \\
\frac{d \rho_{31'}}{d t}  &= i\omega_{1'3}\rho_{31'} -i \Omega_{1'3}(\rho_{1'1'} - \rho_{33}) -i\frac{\mu_{32}E}{\hbar}\rho_{21'} -\gamma_{31'}\rho_{31'} \\
\frac{d \rho_{23}}{d t}   &= i\omega_{32}\rho_{23} -i \frac{\mu_{32}E}{\hbar}(\rho_{33}-\rho_{22}) +i\Omega_{1'3}\rho_{21'} - \gamma_{23}\rho_{23} \\
\frac{d \rho_{21'}}{d t}  &= i\omega_{1'2}\rho_{21'} -i\frac{\mu_{32}E}{\hbar}\rho_{31'} +i\Omega_{1'3}\rho_{23} -\gamma_{21'}\rho_{21'} 
\label{eq:mblochbase-end}
\end{align}

\section{Rotating Wave Approximation}

To simplify the calculations involved we assume the rotating wave and slowly varying amplitude approximations. For the RWA we set: 
\begin{align}
  E(x,t) &= \frac{1}{2} \big ( f(x,t) e^{i(\omega t - kx)} + f^*(x,t)   e^{-i(\omega t - kx)} \big ) \nonumber \\ 
  \rho_{23}(x,t)  &= \eta_{23}(x,t) e^{i(\omega t - kx) } \nonumber \\
  \rho_{21'}(x,t) &= \eta_{21'}(x,t) e^{i(\omega t - kx) }, \\
\end{align}
in order to separate out the slowly varying parts of the electric field and the coherence terms. Substituting into the electric field and DM equations, and 
eliminating the fast oscillating components we get: 

\begin{align}
\partial_{x}f &+\frac{n}{c}\partial_t f = i\frac{N \Gamma \mu_{32} k}{\epsilon_0 n^2} \eta_{23} - \frac{l_0}{2} f \label{eq:rtwaveq} \\
\frac{d \rho_{1'1'}}{d t} &= i\Omega_{1'3} (\rho_{1'3} - \rho_{31'}) + (\frac{1}{\tau_{31'}} + \frac{1}{\tau_{31}})\rho_{33} \nonumber \\ 
& + (\frac{1}{\tau_{21'}} + \frac{1}{\tau_{21}})\rho_{22} - (\frac{1}{\tau_{1'3}} + \frac{1}{\tau_{13}} + \frac{1}{\tau_{1'2}} + \frac{1}{\tau_{12}})\rho_{1'1'} \\
\frac{d \rho_{33}}{d t}   &= i\Omega_{1'3} (\rho_{31'} - \rho_{1'3}) + i\frac{\mu_{32} }{2 \hbar} (f \eta_{23}^*-f^*\eta_{23})+ (\frac{1}{\tau_{1'3}} + \frac{1}{\tau_{13}})\rho_{1'1'} \nonumber \\ 
& +  \frac{1}{\tau_{23}}\rho_{22} - (\frac{1}{\tau_{31'}} + \frac{1}{\tau_{32}} + \frac{1}{\tau_{31}}) \rho_{33} \\
\frac{d \rho_{22}}{d t}   &= -i\frac{\mu_{32} }{2 \hbar} (f \eta_{23}^*-f^*\eta_{23}) + (\frac{1}{\tau_{1'2}} + \frac{1}{\tau_{12}})\rho_{1'1'} \nonumber \\ 
& +  \frac{1}{\tau_{32}}\rho_{33} - (\frac{1}{\tau_{21'}} + \frac{1}{\tau_{23}} + \frac{1}{\tau_{21}}) \rho_{22} \\
\frac{d \rho_{31'}}{d t}  &= i\omega_{1'3}\rho_{31'} -i \Omega_{1'3}(\rho_{1'1'} - \rho_{33}) -i\frac{\mu_{32}}{2\hbar}f^*\eta_{21'} -\gamma_{31'}\rho_{31'} \\
\frac{d \eta_{23}}{d t}   &= -i(\omega - \omega_{32})\eta_{23} -i \frac{\mu_{32}}{2\hbar}f(\rho_{33}-\rho_{22}) +i\Omega_{1'3}\eta_{21'} - \gamma_{23}\eta_{23} \\
\frac{d \eta_{21'}}{d t}  &= -i(\omega - \omega_{1'2})\eta_{21'} -i\frac{\mu_{32}}{2\hbar}f\rho_{31'} +i\Omega_{1'3}\eta_{23} -\gamma_{21'}\eta_{21'} 
\end{align}

\section{Gain calculation}
\label{sec:gain-calculation}
Let us denote the average steady state population density of levels $1'$, $3$ and $2$ as $n_{1'}$, $n_3$ and $n_2$, respectively.  Then we are interested in the steady 
state value for the polarization term $\eta_23$.  Before we introduce the expression for the optical gain, it will be convenient to introduce some simplifying 
notation:
\begin{align}
 &\Gamma_{23}(\omega) = (\omega - \omega_{32}) - i\gamma_{23} \nonumber \\
 &\Gamma_{21'}(\omega) = (\omega - \omega_{1'2}) - i\gamma_{21'} \nonumber \\
 &\Gamma_{31'} = \omega_{1'3} - i\gamma_{31'} \nonumber \\
 &A(\omega) = 1+ \frac{|\mu_{32}f/2\hbar|^2}{\Gamma_{21'}(\omega)\Gamma_{31'}} \nonumber \\
 &B(\omega) = \frac{\Omega_{1'3}}{\Gamma_{21'}(\omega)} \nonumber \\
 &D(\omega) = 1-\frac{\Omega_{1'3}}{\Gamma_{23}(\omega)}\times \frac{B(\omega)}{A(\omega)}
\end{align}

Now solving the three coupled equations:  $\frac{d \rho_{31'}}{d t} = 0 $ , $\frac{d \eta_{21'}}{d t} = 0$ and $\frac{d \eta_{23}}{d t} = 0$ we find that:
\begin{align}
 \bar{\eta}_{23} = -\frac{1}{D(\omega)} \times \Bigg( \frac{ (n_3-n_2)}{\Gamma_{23}(\omega)} + \frac{(n_{1'} - n_3)\times \Omega_{1'3}^2 } 
 {\Gamma_{23}(\omega)\Gamma_{21'}(\omega)\Gamma_{31'} A(\omega) } \Bigg) \times \frac{\mu_{32} f}{2\hbar} .
\end{align}

Now taking Eq. (\ref{eq:rtwaveq}) multiplying it by the conjugate of $f$, i.e. $f^*$,  and adding it to the complex conjugate of the same 
equation times $f$, we get:
\begin{align}
 \partial_{x} I &+\frac{n}{c}\partial_t I = i\frac{N \Gamma \mu_{32} k}{\epsilon_0 n^2} (\eta_{23}f^* - \eta_{23}^* f) - l_0 I \label{eq:rtintensity}, 
\end{align}
where $I = |f|^2$, and plugging in the steady state expression for $\bar{\eta}_{23}$:

\begin{align}
 \partial_{x} I &+\frac{n}{c}\partial_t I = -i\frac{N \Gamma \mu_{32}^2 k}{\epsilon_0 n^2 2 \hbar  } (g(\omega) - g(\omega)^*)I - l_0 I \label{eq:rtintensity}, 
\end{align}
where 
\begin{align}
g(\omega) =  +\frac{1}{D(\omega)} \times \Bigg( \frac{ (n_3-n_2)}{\Gamma_{23}(\omega)} + \frac{(n_{1'} - n_3)\times \Omega_{1'3}^2 } 
 {\Gamma_{23}(\omega)\Gamma_{21'}(\omega)\Gamma_{31'} A(\omega) } \Bigg). \label{eq:gainexpression}
\end{align}
 Now the gain coefficient is equal to  
 $ q(\omega) =   \frac{N \Gamma \mu_{32}^2 k}{\epsilon_0 n^2  \hbar  } \times \Im{(g(\omega))} $ !
 
 Normalizing:
  \begin{align}
  & f = \frac{\mu_{32}}{\hbar} f, \hspace{10mm}
  &\rho_{11} = \frac{N \Gamma k \mu_{32}^2}{\epsilon_0 n^2 \hbar} \rho_{11}, \nonumber \\
 &\rho_{22} = \frac{N \Gamma k \mu_{32}^2}{\epsilon_0 n^2 \hbar} \rho_{22}, \hspace{10mm}
 &\rho_{33} = \frac{N \Gamma k \mu_{32}^2}{\epsilon_0 n^2 \hbar} \rho_{33}, \nonumber \\
  &\eta_{21'} = \frac{N \Gamma k \mu_{32}^2}{\epsilon_0 n^2 \hbar} \eta_{21'}, \hspace{10mm} 
  &\eta_{32} = \frac{N \Gamma k \mu_{32}^2}{\epsilon_0 n^2 \hbar} \eta_{32}, \nonumber \\  
 &\rho_{1'3} = \frac{N \Gamma k \mu_{32}^2}{\epsilon_0 n^2 \hbar} \rho_{31'}, \label{eq:normalization}
 \end{align}
 we see that our gain coefficient can be simply expressed as:
 \begin{align}
   q(\omega) = \Im{(g(\omega))}\\ 
 \end{align}
where all quantities in the expression for $g(\omega)$ (Eq.  \ref{eq:gainexpression}) are taken normalized according to Eq. \ref{eq:normalization}.

\subsection{Weak field limit}

The shape of the steady state spectral gain, Eq.(\ref{eq:gainexpression}), is a complex expression that depends on a number of simulation parameters, such as the 
anticrossing frequency $\Omega_{1'3}$, the frequency difference between all three pairs of levels, their respective dephasing times, 
as well as on the intracavity field intensity $|f|^2$ (see the expression for $A(\omega)$).
A series of simplifying approximations, however, can help us to get a better understanding of the different contributions 
to the gain and hopefully give us a greater insight for the physical mechanisms that contribute to spectral broadening. 

We will start off by assuming that the 
system is biased in such a way that it is close to 1'-3 resonance, i.e. $E_{1'3} = E_{1'} - E_{3} \approx 0$. In that case(and in the absence of strong "pure" 
dephasing) the inclusion of a resonant tunneling term couples the injector and the upper laser level so that electrons spend approximately equal time
within each level (due to Rabi-oscillations). It is then reasonable to assume, that the population difference $n_{1'}-n_3$ is much smaller than the inversion $n_3 - n_2$. Indeed, simulation
data for systems near resonance show us that that is indeed the case and that contribution of the 3-> 2 transition to the gain is more than 10 times stronger that the contribution 
of the 1'->3 transition. Formally, in what follows we will assume that $n_{1'}-n_3 << n_3-n_2 $ and therefore the gain can be written as:
\begin{equation}
 g(\omega) =  -\frac{1}{D(\omega)} \times \frac{ (n_3-n_2)}{\Gamma_{23}(\omega)} . \label{eq:gainexpression}
\end{equation}

Furthermore, when the emitted power is weak, as in the case of THz QCLs, the rabi frequency of the optical transition , i.e. $\mu_{32}f/\hbar$ is negligible compared
to the other frequency constants and thus we are allowed to expand the expression for teh gain in a taylor series of $f$ around $f = 0$.

One can quickly check that expanded (working with normalized quantities) the term $1/D(\omega)$ has the following form:
\begin{align}
 \frac{1}{D(\omega)\Gamma_{23}} = \frac{   \Gamma_{21'}(\omega)\Gamma_{31'} +\|f/2\|^2 }{
 \Gamma_{23}(\omega) \big(  \Gamma_{21'}(\omega)\Gamma_{31'} +\|f/2\|^2 \big) - \Omega_{1'3}^2\Gamma_{31'}}, 
\end{align}

which after taylor expanding around $\|f/2\| \equiv 0$, reads:

\begin{align}
 \frac{1}{D(\omega)\Gamma_{23}} \approx \frac{ \Gamma_{21'}(\omega) }{ \Gamma_{21'}(\omega)\Gamma_{23}(\omega) - \Omega_{1'3}^2} 
 - \frac{\Omega_{1'3}^2 \times \|f/2\|^2}{  \Gamma_{31'} \big (\Gamma_{23}(\omega) \Gamma_{21'}(\omega) - \Omega_{1'3}^2 \big)^2  }. \label{eq:weakfieldgain} 
\end{align}

From Eq. \ref{eq:weakfieldgain}, we can see that the lineshape function for the 3-2 transition contribution has one intensity independent term, which we will show
that can be approximated by a sum of two lorenzians, and a intensity dependent term that looks like saturable absorption with a specific lineshape. Let us first 
investigate the intensity independent term, i.e.  $\mathcal{L}(\omega) = \frac{ \Gamma_{21'}(\omega) }{ \Gamma_{21'}(\omega)\Gamma_{23}(\omega) - \Omega_{1'3}^2} $. Expressing the $\Gamma_{ij}$'s
as  a difference of $\omega$ and a complex frequency component, i.e. :
\begin{align}
 \Gamma_{21'}(\omega) = \omega - \omega_1 \nonumber \\
 \Gamma_{23}(\omega) = \omega - \omega_2 \nonumber, 
\end{align}

where $\omega_1 = \omega_{1'2} +i\gamma_{21'}$ and $\omega_2 = \omega_{32} +i\gamma_{23}$  , we find that the lineshape has the following expansion:
\begin{align}
 \mathcal{L}(\omega) 
 &= \frac{\omega - \omega_1}{(\omega - \omega_1)(\omega - \omega_2) - \Omega_{1'3}^2}
  = \mathcal{L}_1(\omega) + \mathcal{L}_1(\omega) + d\mathcal{L}(\omega) \nonumber \\
 &= \frac{1}{2(\omega - \omega_+)} + \frac{1}{2(\omega - \omega_-)} + \frac{\omega_2 - \omega_1}{2 (\omega - \omega_+) (\omega - \omega_-)}
\end{align}

where 
\begin{align}
\omega_\pm = \frac{ \omega_1 + \omega_2}{2} \pm \frac{1}{2} \sqrt{(\omega_1 - \omega_2)^2 + 4\Omega_{1'3}^2} 
\end{align}

Close to resonance, $(\omega_1 - \omega_2)^2 = (\omega_{1'3} + i (\gamma_{21'} - \gamma_{23}))^2 << 4\Omega_{1'3}^2$, from which it can clearly be seen that
$\mathcal{L}_1$ and $\mathcal{L}_2$

\section{Fabry-Perrot Cavity} 

To derive the equations for a Fabry-Perrot type resonator, we will assume the following ansatz:

\begin{itemize}
 \item {Electric field ansatz: 
    \begin{align}
	E(x,t) &= \frac{1}{2} \big ( f_{+} e^{i(\omega t-kx)} + f_{-} e^{i(\omega t+kx)} + c.c \big )
    \end{align}}
 \item {Population terms ansatz: 
    \begin{align}
       \rho_{ii}(x,t) = \rho_{ii}^0 + \rho_{ii}^+ e^{2ikx} + \rho_{ii}^-e^{-2ikx}
    \end{align} , where $\rho_{ii}^0 \in \mathbb{R} $  and  $\rho_{ii}^+ = (\rho_{ii}^-)^*$ .  }
  \item {Coherence terms ansatz:
    \begin{align}
      \rho_{32} &= \eta^+e^{i(\omega t - kx)} + \eta^-e^{i(\omega t + kx)} \nonumber \\
      \rho_{21'} &= \sigma^+e^{i(\omega t - kx)} + \sigma^-e^{i(\omega t + kx)} \nonumber \\
      \rho_{31'} &= \rho_{31'}^0 + \rho_{31'}^+ e^{2ikx} +  \rho_{31'}^{-} e^{-2ikx} 
    \end{align}.
    }
\end{itemize}

In the following framework, using the RWA, the Maxwell-Bloch equations (\ref{eq:mblochbase-start} - \ref{eq:mblochbase-end}) read: 

\textbf{Populations}
\begin{align}
&\frac{d \rho_{1'1'}}{d t}^{0/\pm} = i\Omega_{1'3} (\rho_{1'3}^{0/\pm} - \rho_{31'}^{0/\pm}) + (\frac{1}{\tau_{31'}} + \frac{1}{\tau_{31}})\rho_{33}^{0/\pm}  \nonumber \\ 
& + (\frac{1}{\tau_{21'}} + \frac{1}{\tau_{21}})\rho_{22}^{0/\pm} - (\frac{1}{\tau_{1'3}} + \frac{1}{\tau_{13}} + \frac{1}{\tau_{1'2}} + \frac{1}{\tau_{12}})\rho_{1'1'}^{0/\pm} \\
&\frac{d \rho_{33}}{d t}^0 = i\Omega_{1'3} (\rho_{31'}^0 - \rho_{1'3}^0) + i\frac{\mu_{32}}{2\hbar} \big (f_+ (\eta^+)^* + f_-(\eta^-)^* -c.c. \big )+ (\frac{1}{\tau_{1'3}} 
+ \frac{1}{\tau_{13}})\rho_{1'1'}^0 \nonumber \\ 
& +  \frac{1}{\tau_{23}}\rho_{22}^0 - (\frac{1}{\tau_{31'}} + \frac{1}{\tau_{32}} + \frac{1}{\tau_{31}}) \rho_{33}^0 \\
&\frac{d \rho_{33}}{d t}^+   = i\Omega_{1'3} (\rho_{31'}^+ - \rho_{1'3}^+) + i\frac{\mu_{32}}{2\hbar}\big (f_- (\eta^+)^* - (f_+)^*\eta^- \big ) 
+ (\frac{1}{\tau_{1'3}} + \frac{1}{\tau_{13}})\rho_{1'1'}^+ \nonumber \\ 
& +  \frac{1}{\tau_{23}}\rho_{22}^+ - (\frac{1}{\tau_{31'}} + \frac{1}{\tau_{32}} + \frac{1}{\tau_{31}}) \rho_{33}^+ \\
&\frac{d \rho_{22}}{d t}^0  = -i\frac{\mu_{32}}{2\hbar} \big (f_+ (\eta^+)^* + f_-(\eta^-)^* -c.c. \big) + (\frac{1}{\tau_{1'2}} + \frac{1}{\tau_{12}})\rho_{1'1'}^0 \\ 
& +  \frac{1}{\tau_{32}}\rho_{33}^0 - (\frac{1}{\tau_{21'}} + \frac{1}{\tau_{23}} + \frac{1}{\tau_{21}}) \rho_{22}^0 \\
&\frac{d \rho_{22}}{d t}^+   = -i\frac{\mu_{32}}{2\hbar} \big (f_- (\eta^+)^* - (f_+)^*\eta^- \big ) + (\frac{1}{\tau_{1'2}} + \frac{1}{\tau_{12}})\rho_{1'1'}^+ \\ 
& +  \frac{1}{\tau_{32}}\rho_{33}^+ - (\frac{1}{\tau_{21'}} + \frac{1}{\tau_{23}} + \frac{1}{\tau_{21}}) \rho_{22}^+ \\
\end{align}

\textbf{Coherence Terms}
\begin{align}
&\frac{d \rho_{31'}}{d t}^0  = i\omega_{1'3}\rho_{31'}^0 -i \Omega_{1'3}(\rho_{1'1'}^0 - \rho_{33}^0) -i\frac{\mu_{32}}{2 \hbar}\big ((f_+)^*\sigma^+ 
+ (f_-)^*\sigma^- \big ) -\gamma_{31'}\rho_{31'}^0 \\
&\frac{d \rho_{31'}}{d t} ^\pm = i\omega_{1'3}\rho_{31'}^\pm  -i\Omega_{1'3}(\rho_{1'1'}^\pm - \rho_{33}^\pm) -i \frac{\mu_{32}}{2 \hbar} (f_{\pm})^* \sigma^{\mp} 
- \gamma_{31'} \rho_{31'}^{\pm}\\
&\frac{d \eta^\pm}{d t}   = -i(\omega - \omega_{32})\eta^\pm -i \frac{\mu_{32}}{2\hbar}(f_\pm \Delta\rho_{32}^0 + f_\mp \Delta\rho_{32}^\mp) +  i\Omega_{1'3}\sigma^{\pm}
- \gamma_{23}\eta^\pm \\
&\frac{d \sigma^\pm}{d t}  = -i(\omega - \omega_{1'2})\sigma^\pm -i \frac{\mu_{32}}{2\hbar}(f_\pm \rho_{31'}^0 + f_\mp \rho_{31'}^\mp) +  i\Omega_{1'3}\eta^{\pm}
- \gamma_{21'}\sigma^\pm
\end{align}


\begin{thebibliography}{1}
\bibitem{callebaut2005} Hans Callebaut and Qing Hu, \emph{Importance of coherence for electron transport in terahertz quantum cascade lasers}, Journal of Applied Physics 98, 104505 (2005)
\bibitem{kumar2009} Sushil Kumar and Qing Hu, \emph{Coherence of resonant-tunneling transport in terahertz quantum-cascade lasers}, Phys. Rev. B 80, 245316 (2009)
\bibitem{christian2014} Christian Jirauschek and Tillmann Kubis, \emph{Modeling techniques for quantum cascade lasers}, Applied Physics Reviews 1, 011307 (2014)
\bibitem{bidegaray2001} B.Bidegaray, A.Bourgeade and D.Reignier, \emph{Introducing Phyisical Relaxation Terms in Bloch Equations}, J. Comp. Phys. 170,603-613 (2001).
\end{thebibliography}




\end{document}
