\documentclass[journal,onecolumn]{IEEEtran}

\renewcommand{\baselinestretch}{1.0} % Change to 1.65 for double spacing

\usepackage{amsmath,amsfonts,amssymb}
\usepackage{graphicx}
\usepackage[colorlinks=true, allcolors=blue]{hyperref}


\usepackage{color}
\usepackage[latin9]{inputenc}
\usepackage{mathrsfs,amsmath}
\usepackage{graphicx}%
\usepackage{float}
\usepackage{amsfonts}%
\usepackage{amssymb}
\usepackage{braket}
\usepackage{bm}
\usepackage{cite}
%necessary for outline section of the article 
%\usepackage{outline}

\newcommand{\mb}[1]{\bm{#1}}
\usepackage[T1]{fontenc}
\def\Nabla{\bm{\nabla}}
\def\bm{\mathbf}
\def\curl{\Nabla\times}
\def\div{\Nabla\cdot}
\def\lap{\Delta}
\def\vlap{\Delta}
\def\x{\hat{e}_{x}}
\def\y{\hat{e}_{y}}
\def\z{\hat{e}_{z}}
\def\p{\partial}
\def\h{\hat}
\def\tw{\tilde{\omega}}
\def\gm{\gamma}
\def\om{\omega}
\def\OM{\Omega}
\def\GM{\Gamma}
\def\dw{\delta\omega}
\def\dth{\Delta\theta}
\def\dk{\delta k}
\def\Hdth{\frac{\dth}{2}} %half Delta Theta
\DeclareMathOperator{\Tr}{Tr}


\title{Analysis of operating regimes of terahertz quantum cascade laser frequency combs}
\author{\IEEEauthorblockN{Petar Tzenov\IEEEauthorrefmark{1},
		David Burghoff\IEEEauthorrefmark{2},
		Qing Hu\IEEEauthorrefmark{2}, 
		Christian Jirauschek\IEEEauthorrefmark{1}}
	
	\IEEEauthorblockA{\IEEEauthorrefmark{1}Institute for Nanoelectronics, Technical University of Munich, D-80333 Munich, Germany}
	
	\IEEEauthorblockA{\IEEEauthorrefmark{2}Department of Electrical Engineering and Computer Science, Research Laboratory of Electronics, Massachusetts Institute of Technology, Cambridge, Massachusetts 02139, USA}
	\thanks{Corresponding author: P. Tzenov (email: petar.tzenov@tum.de).}}



\IEEEtitleabstractindextext

\begin{document} 
	\maketitle
	
	
	
\begin{abstract}
In recent years, quantum cascade lasers (QCLs) have shown tremendous potential for the generation of frequency combs in the mid-infrared (midIR) and terahertz (THz) portions of the electromagnetic spectrum. The research community has experienced success both in the theoretical understanding and experimental realization of QCL devices, capable of generating stable and broadband frequency combs. Specifically, it has been pointed out that four wave mixing (FWM) is the main comb formation process and group velocity dispersion (GVD) is the main comb degradation mechanism. As a consequence, special dispersion compensation techniques have been employed, in order to suppress the latter and simultaneously enhance the former processes. Here, we perform a detailed computational analysis of four wave mixing, group velocity dispersion and spatial hole burning (SHB), all known to play a role in QCLs, and show that SHB has a considerable impact on whether the device will operate as a comb or not. We therefore conclude that for a successful implementation of a quantum cascade laser frequency comb, one would need to address this effect as well. 
\end{abstract}

	
	
\end{document} 
