%%%%%%%%%%%%%%%%%%%%%%%%%%%%%%%%%%%%%%%%%%%%%%%%%%%%%%%
%                File: OpEx_temp.tex                  %
%             Created: 2 September 2009               %
%                Updated: 15 May 2015                 %
%                                                     %
%           LaTeX template file for use with          %
%           OSA's journals Optics Express,            %
%             Biomedical Optics Express,              %
%            and Optical Materials Express            %
%                                                     %
%  send comments to Theresa Miller, tmiller@osa.org   %
%                                                     %
% This file requires style file, opex3.sty, under     %
%              the LaTeX article class                %
%                                                     %
%   \documentclass[10pt,letterpaper]{article}         %
%   \usepackage{opex3}                                %
%                                                     %
%                                                     %
%       (c) 2015 Optical Society of America           %
%%%%%%%%%%%%%%%%%%%%%%%%%%%%%%%%%%%%%%%%%%%%%%%%%%%%%%%

%%%%%%%%%%%%%%%%%%%%%%% preamble %%%%%%%%%%%%%%%%%%%%%%%%%%%
\documentclass[10pt,letterpaper]{article}

\usepackage{color}
\usepackage[latin9]{inputenc}
\usepackage{amsmath}
\usepackage{graphicx}%
\usepackage{float}
\usepackage{amsfonts}%
\usepackage{amssymb}
\usepackage{braket}
\usepackage{bm}
\newcommand{\mb}[1]{\bm{#1}}
\usepackage[T1]{fontenc}



\def\Nabla{\bm{\nabla}}
\def\bm{\mathbf}
\def\curl{\Nabla\times}
\def\div{\Nabla\cdot}
\def\lap{\Delta}
\def\vlap{\Delta}
\def\x{\hat{e}_{x}}
\def\y{\hat{e}_{y}}
\def\z{\hat{e}_{z}}
\def\p{\partial}
\def\uline{\underline}
\def\tw{\tilde{\omega}}
\def\gm{\gamma}
\def\om{\omega}
\def\OM{\Omega}
\def\GM{\Gamma}
\def\dw{\delta\omega}
\def\dth{\Delta\theta}
\def\dk{\delta k}
\def\Hdth{\frac{\dth}{2}} %half Delta Theta

%%%%%%%%%%%%%%%%%%%%%%% begin %%%%%%%%%%%%%%%%%%%%%%%%%%%%%%
\begin{document}
	
	%%%%%%%%%%%%%%%%%% title page information %%%%%%%%%%%%%%%%%%
	\title{Simple template for authors submitting to OSA Express Journals}
	
	
	%%%%%%%%%%%%%%%%%%% abstract and OCIS codes %%%%%%%%%%%%%%%%
	%% [use \begin{abstract*}...\end{abstract*} if exempt from copyright]
	
	%%%%%%%%%%%%%%%%%%%%%%% References %%%%%%%%%%%%%%%%%%%%%%%%%
	\begin{thebibliography}{99}
		
	\end{thebibliography}
	
	%%%%%%%%%%%%%%%%%%%%%%%%%%  body  %%%%%%%%%%%%%%%%%%%%%%%%%%
	\section{Dispersion analysis:MBloch in RWA + SVEA. Ring cavity.}
	
	EOM:
	\begin{subequations}
		\label{eq:threelevelmodel}
		\begin{align}
		&\frac{n}{c}\partial_t f + \partial_{x}f = i\frac{N \Gamma \mu_{ba} k_c}{\epsilon_0 n^2} \eta_{ba} - \frac{l_0}{2} f \label{eq:rtwave} .\\
		&\frac{d \rho_{cc}}{d t} = i\Omega_{cb} (\rho_{cb} - \rho_{bc}) + (\frac{1}{\tau_{bc}} + \frac{1}{\tau_{bd}})\rho_{bb} 
		+ (\frac{1}{\tau_{ac}} + \frac{1}{\tau_{21}})\rho_{aa} - \frac{\rho_{cc}}{\tau_{c}} \\
		&\frac{d \rho_{bb}}{d t} = i\Omega_{cb} (\rho_{bc} - \rho_{cb}) + i\frac{\mu_{ba}}{2\hbar} \big (f^*\eta_{ba}- c.c. \big )+ \frac{1}{\tau_{cb}}\rho_{cc} +  \frac{1}{\tau_{ab}}\rho_{aa} - \frac{\rho_{bb}}{\tau_{3}}  \\
		&\frac{d \rho_{aa}}{d t}  = -i\frac{\mu_{ba}}{2\hbar} \big (f^*\eta_{ba} - c.c. \big ) + \frac{1}{\tau_{ca}}\rho_{cc}  +  \frac{1}{\tau_{ba}}\rho_{bb} - \frac{\rho_{aa}}{\tau_{21}} , \\
		&\frac{d \rho_{cb}}{d t}  = -i\epsilon\rho_{cb} +i \Omega_{cb}(\rho_{cc} - \rho_{bb}) +i\frac{\mu_{ba}}{2 \hbar}f^*\eta_{ca}-\tau_{\parallel cb}^{-1} \rho_{cb},  \\
		&\frac{d \eta_{ba}}{d t}   = i(\omega - \omega_{ba})\eta_{ba} +i \frac{\mu_{ba}}{2\hbar}f(\rho_{bb}-\rho_{aa}) - i\Omega_{cb}\eta_{ca} - \tau_{\parallel ba}^{-1}\eta_{ba}, \\
		&\frac{d \eta_{ca}}{d t}  = i(\omega - \omega_{ba}-\epsilon)\eta_{ca} +i \frac{\mu_{ba}}{2\hbar}f\rho_{cb} - i\Omega_{cb}\eta_{ba} - \tau_{\parallel ca}^{-1}\eta_{ca}.
		\end{align}
	\end{subequations}
	
	
	Steady state solution for the cohrences:
	\begin{align}
	&\frac{d \rho_{cb}}{d t}  = 0 =  -i\epsilon\rho_{cb} +i \Omega_{cb}(\rho_{cc} - \rho_{bb}) +i\frac{\mu_{ba}}{2 \hbar}f^*\eta_{ca}-\tau_{\parallel cb}^{-1} \rho_{cb},  \\
	&\Rightarrow \nonumber \\
	&\rho_{cb} (\tau_{\parallel cb}^{-1}+i\epsilon\rho_{cb}) = i \Omega_{cb}(\rho_{cc} - \rho_{bb}) +i\frac{\mu_{ba}}{2 \hbar}f^*\eta_{ca} \\
	&\Rightarrow \nonumber \\
	&\rho_{cb}  =\frac{1}{(\epsilon\rho_{cb}-i\tau_{\parallel cb}^{-1})} \left( \Omega_{cb}(\rho_{cc} - \rho_{bb}) +\frac{\mu_{ba}}{2 \hbar}f^*\eta_{ca} \right).
	\end{align}
	Now for the $c-b$ coherence:
	\begin{align}
	&\frac{d \eta_{ca}}{d t}  = 0 = i(\omega - \omega_{ba}-\epsilon)\eta_{ca} +i \frac{\mu_{ba}}{2\hbar}f\rho_{cb} - i\Omega_{cb}\eta_{ba} - \tau_{\parallel ca}^{-1}\eta_{ca}. \\
	&\Rightarrow \nonumber \\
	&\eta_{ca}(\tau_{\parallel ca}^{-1} -i(\omega - \omega_{ba}-\epsilon)) =  +i \frac{\mu_{ba}}{2\hbar}f\rho_{cb} - i\Omega_{cb}\eta_{ba}  \\
	&\Rightarrow \nonumber \\
	&\eta_{ca}=  \frac{1}{((\omega - \omega_{ba}-\epsilon)+i\tau_{\parallel ca}^{-1} ) } \left(\Omega_{cb}\eta_{ba} - \frac{\mu_{ba}}{2\hbar}f\rho_{cb} \right)  \\
	&\eta_{ca}=  \frac{1}{((\omega - \omega_{ba}-\epsilon)+i\tau_{\parallel ca}^{-1} ) } \left(\Omega_{cb}\eta_{ba} - \frac{\mu_{ba}}{2\hbar}f\frac{1}{(\epsilon\rho_{cb}-i\tau_{\parallel cb}^{-1})} \left( \Omega_{cb}(\rho_{cc} - \rho_{bb}) +\frac{\mu_{ba}}{2 \hbar}f^*\eta_{ca} \right) \right)
	\end{align}
	Setting
	\begin{align}
	\Gamma_{cb} = (\epsilon\rho_{cb}-i\tau_{\parallel cb}^{-1}) \\
	\Gamma_{ca}(\omega) = ((\omega - \omega_{ba}-\epsilon)+i\tau_{\parallel ca}^{-1} ),
	\end{align}
	we can write:
	\begin{align}
	&\eta_{ca}=  \frac{1}{ \Gamma_{ca} } \times \Omega_{cb}\eta_{ba} - \frac{1}{\Gamma_{ca}(\omega)\Gamma_{cb}}\times \frac{\mu_{ba}}{2 \hbar}f\times \Omega_{cb}(\rho_{cc} - \rho_{bb})- \frac{1}{\Gamma_{ca}(\omega)\Gamma_{cb}}\times|\frac{\mu_{ba}}{2 \hbar}f|^2\times\eta_{ca}
	\end{align}
	or also:
	\begin{align}
	&\eta_{ca}\left(1+\frac{1}{\Gamma_{ca}(\omega)\Gamma_{cb}}\times|\frac{\mu_{ba}}{2 \hbar}f|^2 \right)=  \frac{1}{ \Gamma_{ca} } \times \Omega_{cb}\eta_{ba} - \frac{1}{\Gamma_{ca}(\omega)\Gamma_{cb}}\times \frac{\mu_{ba}}{2 \hbar}f\times \Omega_{cb}(\rho_{cc} - \rho_{bb}) \\ 
	&\Rightarrow \\
	&\eta_{ca}= \left(1+\frac{1}{\Gamma_{ca}(\omega)\Gamma_{cb}}\times|\frac{\mu_{ba}}{2 \hbar}f|^2 \right)^{-1} \left\{ \frac{1}{ \Gamma_{ca} } \times \Omega_{cb}\eta_{ba} - \frac{1}{\Gamma_{ca}(\omega)\Gamma_{cb}}\times \frac{\mu_{ba}}{2 \hbar}f\times \Omega_{cb}(\rho_{cc} - \rho_{bb})\right\} 
	\end{align}
	Again abbreviating:
	\begin{align}
	& A(\omega) = \frac{\Gamma_{ca}(\omega)\Gamma_{cb}}{\Gamma_{ca}(\omega)\Gamma_{cb}+\Omega_{ph}^2},
	\end{align}
	where $\Omega_{ph}$ is the photon induced Rabi frequency, it then follows that:
	\begin{align}
	&\eta_{ca}=  \frac{A(\omega)}{ \Gamma_{ca} } \times \Omega_{cb}\eta_{ba} - \frac{A(\omega)}{\Gamma_{ca}(\omega)\Gamma_{cb}}\times \Omega_{ph}\times \Omega_{cb}(\rho_{cc} - \rho_{bb})
	\end{align}
	Plugging into the equation for $\eta_{ba}$ it therefore follows that:
	\begin{align}
	&\frac{d \eta_{ba}}{d t}   = 0 = i(\omega - \omega_{ba})\eta_{ba} +i \Omega_{ph}(\rho_{bb}-\rho_{aa}) - i\Omega_{cb}\eta_{ca} - \tau_{\parallel ba}^{-1}\eta_{ba}, \\
	&\Rightarrow \nonumber \\
	&\eta_{ba} (\tau_{\parallel ba}^{-1}-i(\omega - \omega_{ba})) = i\Omega_{ph}(\rho_{bb}-\rho_{aa}) - i\Omega_{cb}\eta_{ca} \\ 
	&\Rightarrow \nonumber \\
	&\eta_{ba} = \frac{1}{\Gamma_{ba}(\omega) }  \left(\Omega_{cb}\eta_{ca}-\Omega_{ph}(\rho_{bb}-\rho_{aa}) \right), \\ 
	&\eta_{ba} = \frac{1}{\Gamma_{ba}(\omega) }  \left(\Omega_{cb}\left(\frac{A(\omega)}{ \Gamma_{ca} } \times \Omega_{cb}\eta_{ba} - \frac{A(\omega)}{\Gamma_{ca}(\omega)\Gamma_{cb}}\times \Omega_{ph}\times \Omega_{cb}(\rho_{cc} - \rho_{bb})\right)-\Omega_{ph}(\rho_{bb}-\rho_{aa}) \right), \\ 
	&\eta_{ba}(1-\frac{1}{\Gamma_{ba}\Gamma_{ca}}\times\Omega_{cb}^2 \times A(\omega)) = -\left [ \frac{A}{\Gamma_{ca}\Gamma_{ba}\Gamma_{cb}}\times \Omega_{cb}^2\times(\rho_{cc}-\rho_{bb}) +\frac{\rho_{bb}-\rho_{aa}}{\Gamma_{ba}} \right]\times\Omega_{ph}
	\end{align}
	where $  \Gamma_{ba}(\omega) = ((\omega - \omega_{ba})+i\tau_{\parallel ba}^{-1}). $ Finally:
	\begin{align}
	&\eta_{ba} =-(1-\frac{\Omega_{cb}^2  A(\omega)}{\Gamma_{ba}\Gamma_{ca}})^{-1} \left [ \frac{A}{\Gamma_{ca}\Gamma_{ba}\Gamma_{cb}}\times \Omega_{cb}^2\times(\rho_{cc}-\rho_{bb}) +\frac{\rho_{bb}-\rho_{aa}}{\Gamma_{ba}} \right]\times\Omega_{ph}
	\end{align}
	
	Susceptibility:
	
	Since we have:
	$$
	P(t,x=0) = \epsilon_0\chi E(t) = \epsilon_0 \chi \frac{1}{2} ( f(t)e^{-i\omega t} +f^*(t)e^{+i\omega t}= N\Gamma\mu_{ba}(\eta_{ba}e^{-i\omega t}+\eta_{ba}^*e^{i\omega t}),
	$$
	it therefore follows that 
	$$
	\epsilon_0 \chi = 2N\Gamma\mu_{ba}\eta_{ba}/f(t) = -N\Gamma\frac{\mu_{ba}^2}{\hbar}(1-\frac{\Omega_{cb}^2  A(\omega)}{\Gamma_{ba}\Gamma_{ca}})^{-1} \left [ \frac{A}{\Gamma_{ca}\Gamma_{ba}\Gamma_{cb}}\times \Omega_{cb}^2\times(\rho_{cc}-\rho_{bb}) +\frac{\rho_{bb}-\rho_{aa}}{\Gamma_{ba}} \right]
	$$
	Adding a background polarization $\chi_0 = n^2-1$, the total susceptibility is:
	$$
	\chi = n^2-1+N\Gamma\frac{\mu_{ba}^2}{\hbar\epsilon_0}G_{3lvl}(\omega),
	$$
	where $G_{3lvl}$ has the form:
	$$
	G_{3lvl}(\omega) = -
	(1-\frac{\Omega_{cb}^2  A(\omega)}{\Gamma_{ba}(\omega)\Gamma_{ca}(\omega)})^{-1} \left [ \frac{A(\omega)}{\Gamma_{ca}(\omega)\Gamma_{ba}(\omega)\Gamma_{cb}}\times \Omega_{cb}^2\times(\rho_{cc}-\rho_{bb}) +\frac{\rho_{bb}-\rho_{aa}}{\Gamma_{ba}(\omega)} \right].
	$$
	If the optical field is weak, we get $A(\omega) \approx 1$ which gives 
	\begin{align}
	G_{3lvl}(\omega)&= 
	\frac{\Gamma_{ba}(\omega)\Gamma_{ca}(\omega) }{\Omega_{cb}^2-\Gamma_{ba}(\omega)\Gamma_{ca}(\omega)} \left [ \frac{\Omega_{cb}^2}{\Gamma_{ca}(\omega)\Gamma_{ba}(\omega)\Gamma_{cb}}\times(\rho_{cc}-\rho_{bb}) +\frac{\rho_{bb}-\rho_{aa}}{\Gamma_{ba}(\omega)} \right] \\
	&=  \frac{\Omega_{cb}^2}{\Gamma_{cb}(\Omega_{cb}^2-\Gamma_{ba}(\omega)\Gamma_{ca}(\omega))}\times(\rho_{cc}-\rho_{bb}) +\frac{\Gamma_{ca}(\omega)}{\Omega_{cb}^2-\Gamma_{ba}(\omega)\Gamma_{ca}(\omega)}\times(\rho_{bb}-\rho_{aa}) 
	\end{align}
	
	At design bias, due to the string coherent coupling between levels $\rho_{cc}$ and $\rho_{bb}$ we get $\rho_{cc} - \rho_{bb}\approx {0}$, thus the lineshape is reduced to:
	
	\begin{align}
	\Lambda_{3lvl}^{(1)}(\omega) &= 
	\frac{\Gamma_{ca}(\omega)}{\Omega_{cb}^2-\Gamma_{ba}(\omega)\Gamma_{ca}(\omega)} 
	\end{align}
	
	Giving us the total linear susceptibility:
	\begin{align}
	\chi_{3lvl}^{(1)} = n^2-1+\frac{\Delta N_{ba}\Gamma\mu_{ba}^2}{\hbar\epsilon_0}\times\frac{(\tilde{\omega}-\epsilon)+i\gamma_{ca}}{\Omega_{cb}^2-(\tilde{\omega}+i\gamma_{ba})(\tilde{\omega}-\epsilon+i\gamma_{ca})}, 
	\end{align}
	where $\tilde{\omega} = \omega - \omega_{ba}$ and $\Delta N_{ba} = N(\rho_{bb}-\rho_{aa})$ is the density of active electrons.  The complex refractive index is therefore:
	$$
	\uline{n} = n+\frac{1}{2n}\times \frac{\Delta N_{ba}\Gamma \mu_{ba}^2}{\hbar \epsilon_0} \Lambda_{3lvl}^{(1)}({\omega}) = n+\frac{3}{4}\times \frac{\Delta N_{ba}\Gamma e^2f_{ba}}{n\epsilon_0\omega_{ba}m_e} \Lambda_{3lvl}^{(1)}({\omega}),
	$$ 
	
	where $f_{ba}$ is the oscillator strength. Defining the "modified" plasma frequency [Khurgin optical buffers] as: 
	$$
	\Omega_p = \frac{3}{4}\times \frac{\Delta N_{ba}\Gamma e^2f_{ba}}{n\epsilon_0\omega_{ba}m_e}, 
	$$
	we get:
	$$
	\uline{n} = n+ \Omega_p \times \frac{(\tilde{\omega}-\epsilon)+i\gamma_{ca}}{\Omega_{cb}^2-(\tilde{\omega}+i\gamma_{ba})(\tilde{\omega}-\epsilon+i\gamma_{ca})}.
	$$ 
	
	\subsection{two level system}
	
	Similarly to Eq. (\ref{eq:threelevelmodel}) the Maxwell-Bloch equations for a two level system can be written as: 
	
	\begin{subequations}
		\label{eq:twolevelmodel}
		\begin{align}
		&\frac{n}{c}\partial_t f + \partial_{x}f = i\frac{N \Gamma \mu_{ab} k_c}{\epsilon_0 n^2} \eta_{ab} - \frac{l_0}{2} f \label{eq:rtwave} .\\
		&\frac{d \Delta n_{ab}}{d t} = + i\frac{\mu_{ab}}{\hbar} \big (f^*\eta_{ab}- c.c. \big )- \frac{\Delta n_{ab}-\Delta n_{ab}^{(eq)} }{T_{1}}  \\
		&\frac{d \eta_{ab}}{d t}   = i(\omega - \omega_{ab})\eta_{ab} +i \frac{\mu_{ab}}{2\hbar}f(\Delta n_{ab}) -  \tau_{\parallel ab}^{-1}\eta_{ab},
		\end{align}
	\end{subequations}
	which following an equivalent procedure as above, yields the complex susceptibility:
	
	\begin{equation}
	\chi_{2lvl}^{(1)} = n^2-1-\Omega_{ab} \times \frac{1}{\omega - \omega_{ab} + i\gamma_{ab}},
	\end{equation}
	where
	$$\Omega_{ab} = \frac{3}{4}\times \frac{\Delta N_{ab}\Gamma e^2f_{ab}}{n\epsilon_0\omega_{ab}m_e},  $$
	as defined before ($\Delta N_{ab} = N \Delta n_{ab}$). If we imagine a mixed active region, consisting of a single $3lvl$ system and $M-$ $2lvl$ systems of the type above, we can write down the total linear susceptibility as:
	\begin{align}
	\label{eq:totalsusceptibility}
	\chi_{total}^{(1)} &= n^2-1 + \Omega_p \underbrace{\frac{(\omega-\omega_{32}-\epsilon)+i\gamma_{1'2}^{-1}}{\Omega_{1'3}^2-(\omega-\omega_{32}+i\gamma_{32})(\omega-\omega_{32}-\epsilon+i\gamma_{1'2})}}_{\Lambda_{3lvl}^{(1)}(\omega)} - \sum_{m=1}^{M} \Omega_{ab}^{m} \frac{1}{\omega - \omega_{ab}^{m} + i\gamma_{ab}^{m}} \nonumber \\
	&= n^2-1 + \Omega_p \underbrace{ \left  (\frac{(\omega-\omega_{32}-\epsilon)+i\gamma_{1'2} }{(\omega-\omega_{32}+i\gamma_{32})(\omega-\omega_{32}-\epsilon+i\gamma_{1'2})-\Omega_{1'3}^2} - \sum_{m=1}^{M} r_{ab}^{m}  \times \frac{1}{\omega - \omega_{ab}^{m} + i\gamma_{ab}^{m}} \right )}_{\chi_{\vec{p}}(\omega)^{(1)}},	
	\end{align}
	where 
	$$
	r_{ab}^{m} = \frac{\Omega_{ab}^{m}}{\Omega_p} = \frac{\Delta n_{ab}^{m} |\mu_{ab}^{m}|}{(\rho_{33}-\rho_{22})|\mu_{32}|}.
	$$
	\noindent
	\textbf{Idea}: find the parameters $\omega_{ab}^{m},\gamma_{ab}^{m}, \Delta n_{ab}^{m-(eq)},\mu_{ab}^{m}$ etc. so that the dispersion is minimized and also that the original gain is minimally affected: 
	\begin{align}
	\min_{\vec{p}=[\omega_{ab}^{m},\gamma_{ab}^{m}, \Delta n_{ab}^{m-(eq)},\mu_{ab}^{m}] \in \bm{P}} & \left ( \lambda \int{\Re\{\chi_{\vec{p}}(\omega)^{(1)}\}^2 d\omega} + (1-\lambda)\int{ \Im\{\chi_{\vec{p}}(\omega)^{(1)} - \Lambda_{3lvl}^{(1)}(\omega)\}^2 d\omega} \right ) \nonumber \\
	\Rightarrow \nonumber \\
	\min_{ {\vec{p}} \in \bm{P}} & \left ( \lambda \|\Re\{\chi_{\vec{p}}^{(1)}\}\|_{L_2}^2  + (1-\lambda)\| \Im\{\chi_{\vec{p}}^{(1)} - \Lambda_{3lvl}^{(1)}\}\|_{L_2}^2   \right )
	\end{align}
	where $\lambda \in [0;1]$ is some weight and $\bm{P}$ is the domain of possible values for the optimization variables,  $\omega_{ab}^{m},\gamma_{ab}^{m}, \Delta n_{ab}^{m-(eq)},\mu_{ab}^{m}$ , which can be geometrically represented as a convex polygon enclosed by intersecting hyperplanes in the $M\times4$ dimensional search space. The constraints can simply be written as:
	$$
	\bm{P} \text{ : }  A\cdot \vec{p} \leq \vec{b},
	$$
	where $\vec{p} = [\omega_{ab}^{1},\gamma_{ab}^{1},\cdots \Delta n_{ab}^{M-(eq)},\mu_{ab}^{M}]^T$ is a parameter vector , A is a matrix and $\vec{b}$ is a vector.  
	
	
\section{FM-AM decomposition}
To xxxx the nature of interaction between the high and low frequency lobe components let us write down the total field envelope in trigonometric form
\begin{align}
	f(x,t) &= f^{\delta} e^{i(\dk x -\dw t)} +  f^{-\delta} e^{-i(\dk x -\dw t)} \nonumber \\
		 & = |f^{\delta}|e^{i(\dk x -\dw t + \dth )} + |f^{-\delta}|e^{-i(\dk x -\dw  t)}, \label{eq:phasor_f_01}
\end{align}
where we have taken the poler representation of the amplitudes, denoted with $I^{\pm} = |f^{\pm\delta}|^2$ the magnitude square of the envelopes, which is proportional to their intensity, with $\Delta \theta$ the relative phase, with $\cos\Psi = |f^{+\delta}|/\left( I^+ + I^-\right)^{1/2}$, and similarly  $\sin\Psi = |f^{-\delta}|/\left( I^{+}+I^{-}\right)^{1/2}$.  We can rewrite Eq. (\ref{eq:phasor_f_01}) into a trigonometric form
\begin{align}
	f(x,t) &= \sqrt{I^+ + I^{-}} e^{-i\frac{\dth}{2}} \Big [\cos\Psi e^{i(\dk x -\dw t + \frac{\dth}{2})} \nonumber \\ 
		   & +  \sin\Psi e^{-i(\dk x -\dw t + \frac{\dth}{2})}\Big ]. \label{eq:phasor_f_02}
\end{align}
At steady state, the periodic continuation of $f$, $f_c(x,t)$ will satisfy $f_c(x+L,t) = f_c(x,t-T_R)$ with $L$ being equal to the length of the cavity and $T_{rt} = L/v_g$ the round trip time, with $v_g$ some group velocity. This condition can be met if $f(x,t)$ depends on both coordinates as $f(x,t) = f(x/v_g-t) = f(\tau)$ where $\tau$ denotes the coordinate in co-moving frame of reference. Assuming this steady state condition and also that $\dk / \dw \approx v_g$, Eq. (\ref{eq:phasor_f_02}) further simplifies into 
\begin{align}
f(x,t) &= \sqrt{I} e^{-i\frac{\dth}{2}} \Big [\cos\Psi(\tau) e^{i(\dw\tau + \frac{\dth}{2})} +  \sin\Psi(\tau) e^{-i(\dw\tau + \frac{\dth}{2})}\Big ], \label{eq:phasor_f_03}
\end{align}
where we have also substituted $\sqrt{I^+ + I^{-}}$ with  $\sqrt{I}$. 
Using the Euler identity we get
\begin{align}
f(x,t) &= \sqrt{I} e^{-i\frac{\dth}{2}} \Big [\cos\Psi(\tau) (\cos (\dw\tau+ \frac{\dth}{2}) + i \sin(\dw\tau + \frac{\dth}{2})) \nonumber \\
&+  \sin\Psi(\tau)(\cos (\dw\tau + \frac{\dth}{2}) - i \sin(\dw\tau + \frac{\dth}{2})) \Big ] \nonumber \\ \label{eq:phasor_f_03}
&=\sqrt{I}e^{-i\frac{\dth}{2}} \Big[ \cos\Psi(\tau)\cos (\dw\tau+ \frac{\dth}{2}) + \sin\Psi(\tau)\cos (\dw\tau + \frac{\dth}{2}) \nonumber \\
&+ i\big(\cos\Psi(\tau) \sin(\dw\tau + \frac{\dth}{2})  - \sin\Psi(\tau)\sin(\dw\tau + \frac{\dth}{2})\big)\Big] \nonumber \\
&= \sqrt{I} e^{-i\frac{\dth}{2}}\Big[(\cos\Psi+\sin\Psi) \cos(\dw\tau+ \frac{\dth}{2})+i(\cos\Psi-\sin\Psi)\sin (\dw\tau+ \frac{\dth}{2}) \Big] \nonumber \\
&= \sqrt{I} (\cos\frac{\dth}{2}-i\sin\frac{\dth}{2})\Big[(\cos\Psi+\sin\Psi) \cos(\dw\tau+ \frac{\dth}{2})+i(\cos\Psi-\sin\Psi)\sin (\dw\tau+ \frac{\dth}{2}) \Big].
\end{align}  
Thus the real and imaginary components of $f$ can be written as
\begin{align}
f_{re} &= \sqrt{I} \Big[\cos\Hdth \big(\cos\Psi+\sin\Psi\big) \cos(\dw\tau+ \Hdth) + \sin\Hdth\big(\cos\Psi-\sin\Psi\big)\sin(\dw\tau+\Hdth) )\Big]. \nonumber \\
 &= \sqrt{I}\cos\Psi\times(\cos\Hdth \cos(\dw\tau+ \Hdth) + \sin\Hdth\sin(\dw\tau+\Hdth) ) \nonumber \\
 &+ \sqrt{I}\sin\Psi\times( \cos\Hdth \cos(\dw\tau+ \Hdth)-\sin\Hdth\sin(\dw\tau+\Hdth))\nonumber \\
 &=\sqrt{I}\cos\Psi\times\cos\dw\tau + \sqrt{I}\times\sin\Psi\cos(\dw\tau+\dth). 
\end{align}  
Similarly the imaginary part is 
\begin{align}
f_{im} &= \sqrt{I} \Big[\cos\Hdth \big(\cos\Psi-\sin\Psi\big) \sin(\dw\tau+ \Hdth) - \sin\Hdth\big(\cos\Psi+\sin\Psi\big)\cos(\dw\tau+\Hdth) )\Big]. \nonumber \\
&= \sqrt{I}\cos\Psi\times(\cos\Hdth \sin(\dw\tau+ \Hdth) - \sin\Hdth\cos(\dw\tau+\Hdth) ) \nonumber \\
&- \sqrt{I}\sin\Psi\times( \cos\Hdth \sin(\dw\tau+ \Hdth)+\sin\Hdth\cos(\dw\tau+\Hdth))\nonumber \\
&=\sqrt{I}\cos\Psi\times\sin\dw\tau - \sqrt{I}\times\sin\Psi\sin(\dw\tau+\dth). 
\end{align}  
Thus we see that the general envelope can be written as
\begin{align}
f(\tau) = f_{re}(\tau) +f_{im}(\tau) = \sqrt{I}\left(\cos\Psi \times f_{\text{AM}} + \sin\Psi \times f_{\text{FM}}\right),  
\end{align}
where 
\begin{align}
f_{\text{AM}} &= \cos\dw\tau+i\sin\dw\tau = e^{i\dw\tau}, \\
f_{\text{FM}} &= \cos(\dw\tau+\dth)-i\sin(\dw\tau+\dth) = e^{-i(\dw\tau+\dth)}, \\
\end{align}

In case I CANNOT neglect the absolute phase of $f^{+\delta}$, i.e. $\theta_+$ then the decomposition becomes
\begin{eqnarray}
f(\tau) = f_{re}(\tau) +f_{im}(\tau) = \sqrt{I}e^{i\theta_+}\left(\cos\Psi \times f_{\text{AM}} + \sin\Psi \times f_{\text{FM}}\right),  
\end{eqnarray}
and thus we get to the initial 
\begin{eqnarray}
f(\tau) = f_{re}(\tau) +f_{im}(\tau) = \sqrt{I}\left(\cos\Psi \times e^{i(\dw\tau+\theta_+)} + \sin\Psi \times e^{-i(\dw\tau+\theta_-)}\right),  
\end{eqnarray}



\end{document}
 
