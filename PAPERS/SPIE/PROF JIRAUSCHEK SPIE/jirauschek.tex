%% This document created by Scientific Word (R) Version 3.5

\documentclass[a4paper]{spie}%
\usepackage{amsmath,amsfonts,amssymb}
\usepackage{graphicx}
\usepackage[colorlinks=true, allcolors=blue]{hyperref}
\usepackage{amsmath}%
\usepackage{amsfonts}%
\usepackage{amssymb}
%TCIDATA{OutputFilter=latex2.dll}
%TCIDATA{LastRevised=Thursday, January 05, 2017 13:58:26}
%TCIDATA{<META NAME="GraphicsSave" CONTENT="32">}
\renewcommand{\baselinestretch}{1.0} 
\title
{Simulation and analysis of quantum cascade lasers}
\author{Christian Jirauschek}
\affil
{Institute for Nanoelectronics, Technical University of Munich (TUM), Arcisstr. 21, D-80333 Munich, Germany}
\authorinfo
{Further author information: (Send correspondence to C.J.)\\C.J.: E-mail: jirauschek@tum.de, Telephone: +49 (0)89 289 25300}
\pagestyle{empty} 
\setcounter{page}{301} 


\begin{document}

\title{Simulation and analysis of quantum cascade lasers}
\maketitle%

%TCIMACRO{\TeXButton{abstract}{\begin{abstract}
%In quantum cascade lasers (QCLs), both the spectral gain characteristics and
%the nonlinear optical properties can be custom-tailored through quantum
%engineering. This enables mid-infrared and terahertz generation, and
%applications such as frequency combs. For the further development of QCL
%sources, suitable modeling approaches are required. Here we review various
%methods, ranging from semiclassical to quantum transport techniques. A focus
%will be on the adequate implementation of carrier-light interaction based on multi-domain simulation approaches to
%describe the actual lasing process and nonlinear optical effects. We
%demonstrate that by combining different models, we can exploit their
%corresponding strengths to significantly improve the performance of the
%simulation tool.
%\end{abstract}%
%}}%
%BeginExpansion
\begin{abstract}
In quantum cascade lasers (QCLs), both the spectral gain characteristics and
the nonlinear optical properties can be custom-tailored through quantum
engineering. This enables mid-infrared and terahertz generation, and
applications such as frequency combs. For the further development of QCL
sources, suitable modeling approaches are required. Here we review various
methods, ranging from semiclassical to quantum transport techniques. A focus
will be on the adequate implementation of carrier-light interaction based on multi-domain simulation approaches to
describe the actual lasing process and nonlinear optical effects. We
demonstrate that by combining different models, we can exploit their
corresponding strengths to significantly improve the performance of the
simulation tool.
\end{abstract}%
%EndExpansion%

%TCIMACRO{\TeXButton{TeX field}{\keywords
%{Quantum cascade laser,terahertz, frequency comb, ensemble Monte Carlo,Maxwell-Bloch,quantum transport,difference frequency generation,density matrix}%
%}}%
%BeginExpansion
\keywords
{Quantum cascade laser,terahertz, frequency comb, ensemble Monte Carlo,Maxwell-Bloch,quantum transport,difference frequency generation,density matrix}%
%EndExpansion

%Include a list of keywords after the abstract 

\section{INTRODUCTION}

\label{sec:intro}
%\label{} allows reference to this section

The quantum cascade laser (QCL) is a unipolar device where the lasing
transitions occur between quantized electron states in the conduction band of
a multi-quantum-well semiconductor structure~\cite{1994Sci...264..553F}. By
periodic repetition of the basic structure, a single electron can generate
multiple photons, increasing the quantum efficiency. Here, the lasing
wavelength does not primarily depend on the used material system, but rather
on the quantum design, i.e., the barrier and well widths. In this way,
arbitrary wavelengths in the mid-infrared (MIR) and terahertz (THz) regions
can be accessed. Even broadband optical gain can be obtained by combining
different quantum-well active region
designs~\cite{gmachl2002ultra,rosch2015octave}. Besides the spectral gain
characteristics, also the nonlinear optical properties of the active region
can be custom-tailored through quantum engineering of the electronic energy
states. The design of giant optical nonlinearities enables applications such
as the generation of frequency combs
\cite{2012Natur.492..229H,burghoff2014terahertz,wienold2014evidence,rosch2015octave}
and ultrashort pulses~\cite{wang2009mode,barbieri2011coherent,revin2016active}%
, or the realization of widely tunable room temperature THz sources by
frequency mixing
\cite{belkin2008room,lu2012widely,vijayraghavan2013broadly,lu2014continuous}.

For the further development of QCL sources, modeling approaches are required
which capture the intricate interplay of the different physical effects
involved. Various modeling techniques have been employed
\cite{jirauschek2014modeling}, ranging from semiclassical approaches for the
carrier transport, such as the ensemble Monte Carlo (EMC) method
\cite{2000ApPhL..76.2265I,2004ApPhL..84..645C,2006ApPhL..88f1119L,2007JAP...101f3101G,2009JAP...105l3102J,2005JAP....97d3702B,ezhov2016influence}%
, to quantum transport models, including the density matrix (DM)
\cite{2001PhRvL..87n6603I,2009PhRvB..79p5322W,iotti2016electronic,lindskog2014comparative}
and nonequilibrium Green's function (NEGF)
\cite{2002PhRvB..66h5326W,2009PhRvB..79s5323K,2009ApPhL..95w1111S,Kubis_assess,Kolek_openQCL,Wacker_JSTQE,grange2015contrasting}
approaches. While the quantum transport approaches consider quantum coherence
effects such as resonant tunneling and are thus more accurate, their
computational load exceeds that of semiclassical methods by
far~\cite{2010PhyE...42.2628M}. We demonstrate that by combining various
modeling techniques, we can take advantage of their corresponding strengths to
significantly improve the reliability and computational efficiency of the
simulation tool. In addition to the carrier transport, also the optical cavity
field dynamics can play an important role. A simulation of the coupled
transport and optical effects can be achieved in a multi-domain approach. As
illustrative examples, we discuss simulation schemes and results for high
temperature THz QCLs as well as QCL-based frequency comb sources.

\section{CARRIER TRANSPORT MODELING TECHNIQUES}

\label{sec:carr}

For quantum well systems, quantized electron states are obtained in the
confinement direction. These so-called subbands are characterized by their
eigenenergies $E_{i}$, where $i$ is the subband index. Additionally, the
kinetic electron motion in the transverse xy-plane is considered by the
transverse wave vector $\mathbf{k}=[k_{x},k_{y}]^{\mathrm{T}}$, where
$\mathrm{T}$\ denotes the transpose. Advanced carrier transport approaches
evaluate the scattering events self-consistently based on the corresponding
Hamiltonian, taking into account the involved subbands as well as the
transverse wave vectors of the initial and final states $\left|
i,\mathbf{k}\right\rangle $ and $\left|  i^{\prime},\mathbf{k}^{\prime
}\right\rangle $, respectively.

\begin{figure}[th]
\begin{center}%
\begin{tabular}
[c]{c}%
\includegraphics[width=17cm]{Record_laser_I(V).eps}%
\end{tabular}
\end{center}
\caption[example]{Current-voltage characteristics of a THz QCL at
$10\,\mathrm{K}$. Simulation results with (+) and without (X) lasing included
are compared to experimental data taken from Ref.$\,$\citenum
{fathololoumi2012terahertz} (---).}%
\label{fig:QCL}%
\end{figure}

As mentioned in Section \ref{sec:intro}, such advanced approaches include
semiclassical methods and quantum transport techniques. The most widely used
semiclassical carrier transport approach for QCLs is EMC, which is based on a
stochastic evaluation of the Boltzmann equation%
\begin{equation}
\mathrm{d}_{t}f_{i\mathbf{k}}=\sum_{i^{\prime},\mathbf{k}^{\prime}}\left(
W_{i^{\prime}\mathbf{k}^{\prime},i\mathbf{k}}f_{i^{\prime}\mathbf{k}^{\prime}%
}-W_{i\mathbf{k},i^{\prime}\mathbf{k}^{\prime}}f_{i\mathbf{k}}\right)
.\label{eq:boltz}%
\end{equation}
The carrier distribution function $f_{i\mathbf{k}}\left(  t\right)  $ gives
the probability of the state $\left|  i,\mathbf{k}\right\rangle $ being
occupied at time $t$. $W_{i\mathbf{k},i^{\prime}\mathbf{k}^{\prime}}=\sum
_{m}W_{i\mathbf{k},i^{\prime}\mathbf{k}^{\prime}}^{\left(  m\right)  }$ are
the scattering rates, where the index $m$ denotes the different scattering
mechanisms. In EMC, these are self-consistently evaluated based on Fermi`s
golden rule. Here we consider electron-electron interaction
\cite{2010JAP...107a3104J} and longitudinal-optical phonon
\cite{ezhov2016influence}, acoustic phonon, impurity and interface roughness
scattering, as well as photon-induced transitions
\cite{2010ApPhL..96a1103J,jirauschek2010monte}. Notably, most implementations
of carrier transport methods neglect the interaction of the electrons with the
optical field in the resonator. However, for the simulation of the actual
lasing action, and also for the modeling of nonlinear optical processes in QCL
structures as discussed in Section \ref{sec:nonl}, an inclusion of the optical
cavity field is required.

In Fig.\thinspace\ref{fig:QCL}, the current-voltage characteristics obtained
by EMC for a high temperature THz QCL are compared to available experimental
data \cite{fathololoumi2012terahertz}. It can clearly be seen from
Fig.\thinspace\ref{fig:QCL} that the omission of lasing in the simulation
leads to a greatly underestimated current density in the lasing region, since
the photon-induced contribution to the electron transport is ignored
\cite{matyas2011photon}.

As mentioned in Section \ref{sec:intro}, EMC and other semiclassical methods
do not account for quantum transport effects such as incoherent tunneling
transport across barriers \cite{2005JAP....98j4505C}, resulting from tunnel
coupling between pairs of states \cite{2009PhRvB..79p5322W}. While EMC is in
good agreement with experiment for most bias points considered in
Fig.\thinspace\ref{fig:QCL}, an artificial current spike emerges at around
$9\,\mathrm{kV}/\mathrm{cm}$ where a narrow anticrossing between two coupled
states occurs. Here, the semiclassical approximation breaks down since
resonant tunneling is dampened by dephasing processes, which are not included
in the semiclassical picture \cite{2005JAP....98j4505C}. While quantum
transport approaches naturally account for such effects, they intrinsically
have a significantly higher numerical complexity, impeding the design and
optimization of QCL structures. A promising way to overcome this problem is to
include certain features of quantum transport into semiclassical approaches,
combining improved reliability and versatility with an acceptable numerical
load. In this context, supplementing EMC by certain elements of the DM
approach has proven to be a suitable strategy
\cite{2005JAP....98j4505C,2012ApPhL.100a1108B,freeman2016self}, since the DM
formalism is closely related to the Boltzmann transport model, Eq.\thinspace
\ref{eq:boltz} \cite{2001PhRvL..87n6603I}. Here, the carrier distribution
function corresponds to the diagonal DM elements, $f_{i\mathbf{k}}%
=\rho_{i\mathbf{k},i\mathbf{k}}$, while the off-diagonal elements
$\rho_{i\mathbf{k},i^{\prime}\mathbf{k}^{\prime}}$ describe quantum
correlations between the states which are not contained in the semiclassical
formalism. In particular, an adequate description of incoherent tunneling can
be achieved by introducing a tight-binding basis and using the DM formalism
for tunneling between coupled states with narrow anticrossings
\cite{2005JAP....98j4505C,2012ApPhL.100a1108B,freeman2016self}. Quantum
mechanical dephasing is in the DM approach typically described by rates, where
often empirical values are used \cite{2005JAP....98j4505C}. To obtain a fully
self-consistent approach, dephasing should be computed based on the scattering
Hamiltonian, rather than using empirical values. Employing this strategy, we
have developed an DM-EMC approach building upon our well-proven EMC simulation
tool, where Ando's theory has been used to model the dephasing rates
\cite{ando1985line,freeman2016self}. In Fig.\thinspace\ref{fig:QCL}, the
obtained simulation result is shown for $9\,\mathrm{kV}/\mathrm{cm}$,
demonstrating that DM-EMC avoids artificial current spikes by correctly
describing incoherent tunneling transport. A further deviation of EMC from the
experimental data is observed for low biases, where the simulation
underestimates the current density. In this context, the implementation of
Lorentzian broadening of the energy states has been discussed as another
example of DM-based corrections to the semiclassical EMC method
\cite{2013ApPhL.102a1101M}.

An alternative strategy aiming at accuracy and numerical efficiency, which has
also been successfully used for the simulation of QCLs, is to start with a
general quantum transport method such as NEGF and introduce simplifications to
reduce the computational load. This includes for example the constant $k$
approximation which neglects the dependence of the scattering self-energies on
the in-plane wave vector \cite{Kubis_assess,Wacker_JSTQE}, and the replacement
of self-energies by quasi-equilibrium expressions \cite{greck2015efficient}.

\section{SIMULATION OF QCL STRUCTURES WITH OPTICAL NONLINEARITIES}

\label{sec:nonl}

Recently, the implementation of giant optical nonlinearities into QCL\ designs
has been exploited to obtain novel functionalities. This includes frequency
conversion, which has especially been utilized for the realization of widely
tunable room temperature THz sources
\cite{belkin2008room,lu2012widely,vijayraghavan2013broadly,lu2014continuous}.
Another example is the coherent nonlinear interaction between the optical
field and the electrons, enabling the generation of ultrashort
pulses~\cite{wang2009mode,barbieri2011coherent,revin2016active} as well as
frequency combs
\cite{2012Natur.492..229H,burghoff2014terahertz,wienold2014evidence,rosch2015octave}%
.

\begin{figure}[th]
\begin{center}%
\begin{tabular}
[c]{c}%
\includegraphics[width=17cm]{multidomain.eps}%
\end{tabular}
\end{center}
\caption[example]{Schematic illustration of a multi-domain simulation approach
for QCL frequency comb structures.}%
\label{fig:scheme}%
\end{figure}

A detailed, self-consistent simulation approach for such devices requires
careful modeling of both the electron transport and the optical dynamics. This
is naturally achieved by a multi-domain simulation framework. We have
developed such an approach for the modeling of frequency comb QCL\ structures
\cite{tzenov2016time}, as illustrated in Fig.\thinspace\ref{fig:scheme}. The
carrier transport is here evaluated by an EMC or DM-EMC simulation approach,
coupled to a Schr\"{o}dinger-Poisson solver providing the quantized state
wavefunctions and eigenenergies of the QCL\ design. The coherent nonlinear
interaction between the electrons and the optical cavity field is captured by
the Bloch equations, coupled to Maxwell's equations describing the field
propagation \cite{2007PhRvA..75c1802W}. The Bloch equations include the
electron transport in terms of scattering-induced transition and dephasing
rates, which are extracted from the carrier transport simulations. Besides
capturing the coherent carrier-light interaction, the main advantage of the
Maxwell-Bloch equation system is its modest numerical complexity, allowing for
a full spatiotemporal treatment as required for modeling the frequency comb
dynamics. The computational burden associated with the EMC or DM-EMC approach
impedes spatiotemporally resolved carrier transport simulations. However,
since the transition and dephasing rates can be treated as constant in good
approximation, they can be extracted from stationary carrier transport
simulations using effective resonator parameters. In this way, the main
drawback of the Maxwell-Bloch equation model, i.e., its dependence on
empirical rates, can be avoided, and a fully self-consistent numerical
approach is obtained. This multi-domain simulation strategy can be adapted to
different kinds of QCL structures with optical nonlinearities. For example, a
similar approach has been developed for the modeling of THz difference
frequency generation in QCL\ structures
\cite{jirauschek2013monte,jirauschek2015monte}.

\begin{figure}[th]
\begin{center}%
\begin{tabular}
[c]{c}%
\includegraphics[width=17cm]{comb.eps}%
\end{tabular}
\end{center}
\caption[example]{Optical power spectrum of the QCL frequency comb source: (a)
Experimental data from Ref.$\,$\citenum{burghoff2015evaluating} and (b)
simulation.}%
\label{fig:comb}%
\end{figure}

In Fig.\thinspace\ref{fig:comb}, the experimentally measured power spectrum of
a THz frequency comb QCL source \cite{burghoff2015evaluating} is compared to
simulation data from a multi-domain modeling approach as described above. Good
agreement is found; especially the splitting of the spectrum into a high and
low frequency lobe is well reproduced in the simulation.

\section{CONCLUSION}

\label{sec:concl}

We have discussed advanced techniques for modeling the carrier transport in
QCLs, ranging from semiclassical methods to fully quantum mechanical
approaches. Combining a semiclassical technique such as ensemble Monte Carlo
with certain aspects of quantum transport, as for example described by the
density matrix approach, can yield improved accuracy and reliability while
keeping the numerical load acceptable. For the simulation of QCL operation in
the lasing regime, the inclusion of the optical cavity field is crucial to
obtain good agreement with experiment since the photon-induced carrier
transport can play a significant role. In particular, if we want to model the
actual lasing action or nonlinear optical effects in the QCL\ structure, the
optical field must be considered.

In nonlinear optical QCL devices, the complex carrier-light interaction calls
for a multi-domain simulation approach, treating the carrier transport and
optical cavity field on an equal footing and allowing for self-consistent
simulations without empirical parameters. We have illustrated such an approach
for the modeling of QCL-based frequency comb sources and demonstrated good
agreement with experimental results.

\acknowledgments
%equivalent to \section*{ACKNOWLEDGMENTS}       

This work was supported by the German Research Foundation (DFG) within the
Heisenberg program (JI 115/4-1) and under DFG Grant No. JI 115/9-1.%

%TCIMACRO{\TeXButton{spiebib}{\bibliography{qcl}
%\bibliographystyle{spiebib} }}%
%BeginExpansion
\bibliography{qcl}
\bibliographystyle{spiebib}
%EndExpansion
\end{document}